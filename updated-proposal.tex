\documentclass[12pt]{article}

%auto-ignore
%this ensures the arxiv doesn't try to start TeXing here.
%!TEX root =preproposal.tex

\usepackage{etex}
\usepackage{amsmath,amssymb,amsfonts,amsthm}
\usepackage{ifpdf}
\usepackage{leftidx}

\usepackage{etoolbox}
\usepackage{fullpage}
\usepackage{longtable}
\usepackage{pdflscape}
\usepackage{multicol}
\usepackage{enumerate}
\usepackage[all]{xy}
\input xy
\xyoption{all}

\usepackage{libertine}
\usepackage[T1]{fontenc}

\usepackage[backend=bibtex8,style=alphabetic,doi=false,isbn=false,url=false,minnames=6,maxnames=6]{biblatex}
\setcounter{biburlnumpenalty}{9000}
\setcounter{biburllcpenalty}{1000}
\setcounter{biburlucpenalty}{8000}
\renewbibmacro{in:}{%
  \ifentrytype{article}{}{\printtext{\bibstring{in}\intitlepunct}}}
\renewbibmacro*{volume+number+eid}{%
  \printtext{vol.}
  \printfield{volume}%
%  \setunit*{\adddot}% DELETED
  \setunit*{\addnbspace}% NEW (optional); there's also \addnbthinspace
  \printfield{number}%
  \setunit{\addcomma\space}%
  \printfield{eid}}
\DeclareFieldFormat[article]{number}{\mkbibparens{#1}}

\usepackage{xifthen}  % http://ctan.org/pkg/xifthen
\usepackage{breqn}
\usepackage{yfonts}
\usepackage{afterpage}

\usepackage{xcolor}
\definecolor{dark-red}{rgb}{0.7,0.25,0.25}
\definecolor{dark-blue}{rgb}{0.15,0.15,0.55}
\definecolor{medium-blue}{rgb}{0,0,0.65}
\definecolor{DarkGreen}{RGB}{0,150,0}


\ifpdf
\usepackage[pdftex,plainpages=false,hypertexnames=false,pdfpagelabels,breaklinks]{hyperref}
\else
\usepackage[dvips,plainpages=false,hypertexnames=false,breaklinks]{hyperref}
\fi
\hypersetup{
   colorlinks, linkcolor={purple},
   citecolor={medium-blue}, urlcolor={medium-blue}
}


% Page size %%%%%%%%%%%%%%%%%%%%%%%%%%%%%%%%%%%%%%%%%%%
\setlength\topmargin{0in}
\setlength\headheight{0in}
\setlength\headsep{0in}
\setlength\textheight{9in}
\addtolength{\hoffset}{-0.25in}
\addtolength{\textwidth}{.5in}
\setlength\parindent{0.25in}

\usepackage{tikz}
\usetikzlibrary{calc}
\usetikzlibrary{shapes}
\usetikzlibrary{backgrounds}
\usetikzlibrary{decorations.pathreplacing}
\usepackage{tikz-qtree}

\tikzstyle{shaded}=[fill=red!10!blue!20!gray!30!white]
\tikzstyle{unshaded}=[fill=white]
\tikzstyle{empty box}=[circle, draw, thick, fill=white, opaque, inner sep=2mm]
\tikzstyle{annular}=[scale=.7, inner sep=1mm, baseline]
\tikzstyle{rectangular}=[scale=.75, inner sep=1mm, baseline=-.1cm]



\usepackage{tcolorbox}
\tcbuselibrary{breakable}
\tcbuselibrary{skins}





% ----------------------------------------------------------------
\vfuzz2pt % Don't report over-full v-boxes if over-edge is small
\hfuzz2pt % Don't report over-full h-boxes if over-edge is small
% ----------------------------------------------------------------

% diagrams -------------------------------------------------------
% figures ---------------------------------------------------------
%%% borrowed from Dror's cobordisms paper, use this to include eps or pdf graphics.
\newcommand{\pathtodiagrams}{diagrams/}

\newcommand{\mathfig}[2]{{\hspace{-3pt}\begin{array}{c}%
  \raisebox{-2.5pt}{\includegraphics[width=#1\textwidth]{\pathtodiagrams #2}}%
\end{array}\hspace{-3pt}}}

\newcommand{\inputtikz}[1]{\input{diagrams/tikz/#1.tex}}

\newcommand{\arxiv}[1]{\href{https://arxiv.org/abs/#1}{\small  arXiv:#1}}
\newcommand{\arXiv}[1]{\href{https://arxiv.org/abs/#1}{\small  arXiv:#1}}
\newcommand{\doi}[1]{\href{https://dx.doi.org/#1}{{\small DOI:#1}}}
\newcommand{\euclid}[1]{\href{https://projecteuclid.org/getRecord?id=#1}{{\small  #1}}}
\newcommand{\mathscinet}[1]{\href{https://www.ams.org/mathscinet-getitem?mr=#1}{\small  #1}}
\newcommand{\googlebooks}[1]{(preview at \href{https://books.google.com/books?id=#1}{google books})}
\newcommand{\notfree}{}
\newcommand{\numdam}[1]{}

% THEOREMS -------------------------------------------------------
\theoremstyle{plain}
%\newtheorem*{fact}{Fact}
\newtheorem{prop}{Proposition}[section]
\newtheorem{conj}[prop]{Conjecture}
\newtheorem{thm}[prop]{Theorem}
\newtheorem{thmalpha}{Theorem}
\renewcommand*{\thethmalpha}{\Alph{thmalpha}}
\newtheorem{recipe}[prop]{Recipe}
\newtheorem{lem}[prop]{Lemma}
\newtheorem{cor}[prop]{Corollary}
\newtheorem{fact}[prop]{Fact}
\newtheorem{facts}[prop]{Facts}
\newtheorem*{cor*}{Corollary}
\newtheorem*{thm*}{Theorem}
%\newtheorem*{example}{Example}
\newtheorem{question}{Question}
\newenvironment{rem}{\\ \noindent\textsl{Remark.}}{}  % perhaps looks better than rem above?
\numberwithin{equation}{section}
%\numberwithin{figure}{section}

\theoremstyle{remark}
\newtheorem{example}[prop]{Example}
\newtheorem*{exc}{Exercise}
\newtheorem{remark}[prop]{Remark}           
\newtheorem*{rem*}{Remark}               %unnumbered remark
\newtheorem*{example*}{Example}                %unnumbered exercise

\theoremstyle{definition}
\newtheorem{defn}[prop]{Definition}         % numbered definition
\newtheorem{assumption}[prop]{Assumption}   
\newtheorem{nota}[prop]{Notation}   
\newtheorem*{defn*}{Definition}             % unnumbered definition

% \usepackage{parskip}
% Modifies the spacing above theorem environments, which is messed up when using the parskip package.
% (http://tex.stackexchange.com/questions/22119)
% \makeatletter \def\thm@space@setup{\thm@preskip=\parskip \thm@postskip=0pt} \makeatother

\newenvironment{mycolorbox}[1][]%
  {\if\detokenize{#1}\relax\relax%
      \begin{tcolorbox}%
    \else%
      \begin{tcolorbox}[#1]%
    \fi%
  \vspace{-3mm}%
  \parskip=0.5\baselineskip \advance\parskip by 0pt plus 2pt%
  \parindent=0pt%
}
  {\end{tcolorbox}}

\makeatletter
\@ifpackagelater{tcolorbox}{2015/01/01}%
  {%
    \newenvironment{boxedexample}{\begin{mycolorbox}[breakable,notitle,boxrule=1pt,colback=blue!5,colframe=blue!20,enhanced jigsaw]}{\end{mycolorbox}}
  }
  {%
    \newenvironment{boxedexample}{\begin{mycolorbox}[breakable,notitle,boxrule=1pt,colback=blue!5,colframe=blue!20]}{\end{mycolorbox}}
  }%
\makeatother
\newenvironment{boxedexample*}{\begin{mycolorbox}[notitle,boxrule=1pt,colback=blue!5,colframe=blue!20]}{\end{mycolorbox}}

%\usepackage{stmaryd}
\newcommand{\llbracket}{[\![}
\newcommand{\rrbracket}{]\!]}
\newcommand{\truth}[1]{\llbracket #1 \rrbracket}

\theoremstyle{plain}
\newcommand{\noqed}{\renewcommand{\qedsymbol}{}}

% Marginal notes in draft mode -----------------------------------
\newcommand{\scott}[1]{\stepcounter{comment}{{\color{green} $\star^{(\arabic{comment})}$}}\marginpar{\color{green}  $\star^{(\arabic{comment})}$ \usefont{T1}{scott}{m}{n}  #1 --S}}     % draft mode
\newcounter{comment}
\newcommand{\noop}[1]{}
\newcommand{\todo}[1]{\textcolor{blue}{\textbf{TODO: #1}}}
\newcommand{\nn}[1]{\textcolor{red}{[[#1]]}}

% \mathrlap -- a horizontal \smash--------------------------------
% For comparison, the existing overlap macros:
% \def\llap#1{\hbox to 0pt{\hss#1}}
% \def\rlap#1{\hbox to 0pt{#1\hss}}
\def\clap#1{\hbox to 0pt{\hss#1\hss}}
\def\mathllap{\mathpalette\mathllapinternal}
\def\mathrlap{\mathpalette\mathrlapinternal}
\def\mathclap{\mathpalette\mathclapinternal}
\def\mathllapinternal#1#2{%
\llap{$\mathsurround=0pt#1{#2}$}}
\def\mathrlapinternal#1#2{%
\rlap{$\mathsurround=0pt#1{#2}$}}
\def\mathclapinternal#1#2{%
\clap{$\mathsurround=0pt#1{#2}$}}

% MATH -----------------------------------------------------------
\newcommand{\Natural}{\mathbb N}
\newcommand{\Integer}{\mathbb Z}
\newcommand{\Rational}{\mathbb Q}
\newcommand{\Real}{\mathbb R}
\newcommand{\Complex}{\mathbb C}
\newcommand{\Field}{\mathbb F}

% tricky way to iterate macros over a list
\def\semicolon{;}
\def\applytolist#1{
    \expandafter\def\csname multi#1\endcsname##1{
        \def\multiack{##1}\ifx\multiack\semicolon
            \def\next{\relax}
        \else
            \csname #1\endcsname{##1}
            \def\next{\csname multi#1\endcsname}
        \fi
        \next}
    \csname multi#1\endcsname}

% \def\cA{{\cal A}} for A..Z
\def\calc#1{\expandafter\def\csname c#1\endcsname{{\mathcal #1}}}
\applytolist{calc}QWERTYUIOPLKJHGFDSAZXCVBNM;
% \def\bbA{{\mathbb A}} for A..Z
\def\bbc#1{\expandafter\def\csname bb#1\endcsname{{\mathbb #1}}}
\applytolist{bbc}QWERTYUIOPLKJHGFDSAZXCVBNM;
% \def\bfA{{\mathbf A}} for A..Z
\def\bfc#1{\expandafter\def\csname bf#1\endcsname{{\mathbf #1}}}
\applytolist{bfc}QWERTYUIOPLKJHGFDSAZXCVBNM;


\DeclareMathOperator{\depth}{depth}
\DeclareMathOperator{\nbhd}{nbhd}
\DeclareMathOperator{\Span}{span}
\DeclareMathOperator{\Tr}{Tr}
\DeclareMathOperator{\sh}{sh}
\DeclareMathOperator{\un}{un}
\DeclareMathOperator{\FPdim}{FPdim}
\DeclareMathOperator{\coeff}{coeff}

\newcommand{\set}[2]{\left\{#1\middle|#2\right\}}
\newcommand{\jw}[1]{f^{(#1)}}
\newcommand{\twoone}{{\rm II}$_1$}

\newcommand{\id}{\boldsymbol{1}}
\renewcommand{\imath}{\mathfrak{i}}
\renewcommand{\jmath}{\mathfrak{j}}

\newcommand{\qRing}{\Integer[q,q^{-1}]}
\newcommand{\qMod}{\qRing-\operatorname{Mod}}
\newcommand{\ZMod}{\Integer-\operatorname{Mod}}

\newcommand{\into}{\hookrightarrow}
\newcommand{\onto}{\mapsto}
\newcommand{\iso}{\cong}
\newcommand{\actsOn}{\circlearrowright}
\newcommand{\isoto}{\overset{\iso}{\to}}

\newcommand{\htpy}{\simeq}

\newcommand{\abs}[1]{\left|#1\right|}
\newcommand{\norm}[1]{\left|\left|#1\right|\right|}
\newcommand{\ip}[1]{\left< #1\right>}

\newcommand{\relations}[2]{\left<#1 \;\left| \; #2 \right. \right>}
\newcommand{\pairing}[2]{\left\langle#1 ,#2 \right\rangle}

\newcommand{\code}[1]{{\tt #1}}
\newcommand{\MMA}{\code{Mathematica} {}}

\makeatletter
\newcommand{\hashdef}[2]{\@namedef{#1}{#2}}
\newcommand{\hashlookup}[1]{\@nameuse{#1}}
\makeatother


\newcommand{\card}[1]{\sharp{#1}}

\newcommand{\bdy}{\partial}
\newcommand{\compose}{\circ}
\newcommand{\eset}{\emptyset}
\newcommand{\disj}{\sqcup}

\newcommand{\psmallmatrix}[1]{\left(\begin{smallmatrix} #1 \end{smallmatrix}\right)}

\newcommand{\directSum}{\oplus}
\newcommand{\DirectSum}{\bigoplus}
\newcommand{\tensor}{\otimes}
\newcommand{\Tensor}{\bigotimes}

\newcommand{\db}[1]{\left(\left(#1\right)\right)}

\newcommand{\su}[1]{\mathfrak{su}_{#1}}
\newcommand{\csl}[1]{\mathfrak{sl}_{#1}}
\newcommand{\uqsl}[1]{U_q\left(\csl{#1}\right)}


\newcommand{\Mat}[1]{\operatorname{\mathbf{Mat}}\left(#1\right)}
\newcommand{\Inv}[1]{\operatorname{Inv}\left(#1\right)}
\DeclareMathOperator{\Hom}{Hom}
%\newcommand{\Hom}[3]{\operatorname{Hom_{#1}}\!\!\left(#2 \to #3\right)}
\newcommand{\End}[1]{\operatorname{End}\left(#1\right)}
\newcommand{\im}{\operatorname{im}}
\newcommand{\Aut}{\operatorname{Aut}}
\newcommand{\Irr}{\operatorname{Irr}}
\newcommand{\Gal}{\operatorname{Gal}}

\newcommand{\lk}[2]{\operatorname{lk}\left(#1,#2\right)}
\newcommand{\fr}[1]{\operatorname{fr}\left(#1\right)}

\newcommand{\asbimod}[2]{\operatorname{Mod}'\left(#1 \subset #2\right)}
\newcommand{\sbimod}[2]{\operatorname{Mod}\left(#1 \subset #2\right)}
\newcommand{\abimod}[2]{#1-\operatorname{Mod}'-#2}
\newcommand{\bimod}[2]{#1-\operatorname{Mod}-#2}
\newcommand{\bimodule}[3]{\leftidx{_#1}{#2}{_#3}}

\newcommand{\pa}{\mathcal{PA}}
\newcommand{\TL}{\mathcal{TL}}
\newcommand{\JW}[1]{f^{(#1)}}
\newcommand{\tr}[1]{\text{tr}(#1)}
\newcommand{\dn}[1]{{\mathcal D}{\mathit (#1)}}
\newcommand{\Rep}{{\sf Rep}}
\newcommand{\gA}{{\textgoth{A}}}

\newcommand{\directSumStack}[2]{{\begin{matrix}#1 \\ \DirectSum \\#2\end{matrix}}}
\newcommand{\directSumStackThree}[3]{{\begin{matrix}#1 \\ \DirectSum \\#2 \\ \DirectSum \\#3\end{matrix}}}

\newcommand{\grading}[1]{{\color{blue}\{#1\}}}
\newcommand{\shift}[1]{\left[#1\right]}

\newcommand{\tensorover}[1]{\otimes_{#1}}
\newcommand{\tensorhat}{\widehat{\Tensor}}

\newcommand{\LL}{\mathcal{L}}
\newcommand{\Lhat}{\hat{\mathcal{L}}}
\newcommand{\writhe}{\operatorname{writhe}}

\newenvironment{narrow}[2]{%
\vspace{-0.4cm}% horrible hack, by scott % this only seems to be appropriate in beamer mode...
\begin{list}{}{%
\setlength{\topsep}{0pt}%
\setlength{\leftmargin}{#1}%
\setlength{\rightmargin}{#2}%
\setlength{\listparindent}{\parindent}%
\setlength{\itemindent}{\parindent}%
\setlength{\parsep}{\parskip}}%
\item[]}{\end{list}}

\makeatletter

%% this adds some diagnostic messages in the log file, helpful for tracking down permanent 'labels may have changed' warnings.
\def\@testdef #1#2#3{%
  \def\reserved@a{#3}\expandafter \ifx \csname #1@#2\endcsname
 \reserved@a  \else
\typeout{^^Jlabel #2 changed:^^J%
\meaning\reserved@a^^J%
\expandafter\meaning\csname #1@#2\endcsname^^J}%
\@tempswatrue \fi}


%%%Tikz macro

\newcommand{\roundNbox}[6]{
	\draw[rounded corners=5pt, very thick, #1] ($#2+(-#3,-#3)+(-#4,0)$) rectangle ($#2+(#3,#3)+(#5,0)$);
	\coordinate (ZZa) at ($#2+(-#4,0)$);
	\coordinate (ZZb) at ($#2+(#5,0)$);
	\node at ($1/2*(ZZa)+1/2*(ZZb)$) {#6};
}

\newcommand{\ncircle}[5]{
	\draw[thick, #1] #2 circle (#3);
	\node at #2 {#5};
	\filldraw[red] ($#2+(#4:#3cm)$) circle (.05cm);
%	\node at ($#2+(#4:.15cm)+(#4:#3cm)$) {$\star$};
}

\usetikzlibrary{decorations.pathmorphing}

\pgfdeclaredecoration{complete sines}{initial}
{
    \state{initial}[
        width=+0pt,
        next state=sine,
        persistent precomputation={\pgfmathsetmacro\matchinglength{
            \pgfdecoratedinputsegmentlength / int(\pgfdecoratedinputsegmentlength/\pgfdecorationsegmentlength)}
            \setlength{\pgfdecorationsegmentlength}{\matchinglength pt}
        }] {}
    \state{sine}[width=\pgfdecorationsegmentlength]{
        \pgfpathsine{\pgfpoint{0.25\pgfdecorationsegmentlength}{0.5\pgfdecorationsegmentamplitude}}
        \pgfpathcosine{\pgfpoint{0.25\pgfdecorationsegmentlength}{-0.5\pgfdecorationsegmentamplitude}}
        \pgfpathsine{\pgfpoint{0.25\pgfdecorationsegmentlength}{-0.5\pgfdecorationsegmentamplitude}}
        \pgfpathcosine{\pgfpoint{0.25\pgfdecorationsegmentlength}{0.5\pgfdecorationsegmentamplitude}}
}
    \state{final}{}
}

\title{Proposal for MSRI semester on \textbf{Quantum Symmetries}}
\author{Scott Morrison and Noah Snyder}

\begin{document}
\maketitle

% The sample preproposal is at \url{https://drive.google.com/file/d/1V3-aTmbB8uwoGgk5WAtz1kZ8cymrOPfLvqLozm_Ep0P_kl1ZmXQf6RBTm_6Q0oFnEQoVGxrys3Vs6VuO/view}

% The recent proposal in algebraic topology is at \url{https://drive.google.com/file/d/0B3vGTOtp9YXBUXd5OEhFUV9FRUk/view?usp=sharing}

We are proposing a one-semester MSRI program with the goal of advancing and unifying the study of quantum symmetry throughout mathematics, with an emphasis on the appearance of tensor categories in the study of operator algebras, representation theory, mathematical physics, and topology.

\section{Organizers}
The semester will be organized by:
\begin{itemize}
  \setlength{\itemsep}{1pt}
  \setlength{\parskip}{0pt}
  \setlength{\parsep}{0pt}
\item Vaughan Jones, Vanderbilt University
\item Scott Morrison, Australian National University
\item Victor Ostrik, University of Oregon
\item Emily Peters, Loyola University Chicago
\item Eric Rowell, Texas A\&M University
\item Noah Snyder, Indiana University
\item Chelsea Walton, Temple University
\end{itemize}
All are enthusiastic about the program, and can commit to attending for a substantial part of the semester. We have successfully organized workshops previously with several subsets of this group.

Assuming the program is held in Spring 2020, Scott Morrison, Victor Ostrik,
Eric Rowell,  Noah Snyder, and Chelsea Walton all anticipate being able to
take a sabbatical and to attend the entire program. Vaughan Jones would be able to attend for approximately one month, and Emily Peters for approximately 6 weeks.


\section{What is Quantum Symmetry?}

Symmetry, as formalized by group theory, is ubiquitous across mathematics and science. Classical examples include point groups in crystallography, Noether's theorem relating differentiable symmetries and conserved quantities, and the classification of fundamental particles according to irreducible representations of the Poincar\'e group and the internal symmetry groups of the standard model. However, in some quantum settings, the notion of a group is no longer enough to capture all symmetries.  Important motivating examples include Galois-like symmetries of von Neumann algebras, anyonic particles in condensed matter physics, and deformations of universal enveloping algebras. It is not clear how to directly generalize the notion of a group to encompass these examples.

To achieve this generalization, we first take a different viewpoint on classical symmetries, replacing the group with an appropriate category of representations.  Such a category of representations has a rich structure: most crucially one can form tensor products of representations.  Thus to any group we can naturally assign a tensor category, which captures many of the features of the group. This replacement has a long history in both mathematics (pioneered by Weyl) and physics (where the basic excitations of a system are typically indexed by the irreducible representations of the symmetry group).

When we produce a tensor category as the representation theory of a group, the tensor product will always be commutative. We can consider a more general situation, and thus begin the formalization of all examples of quantum symmetries, as various classes of tensor categories.  This generalization of groups by tensor categories is a natural analogue of generalizing the notion of space in algebraic geometry, by replacing rings of functions on a variety with noncommutative rings, or of thinking of $C^*$-algebras as noncommutative topological spaces.

Quantum symmetry has appeared independently many times in mathematics, in different guises including tensor categories, topological phases of matter, subfactors, Hopf algebras, and quantum groups. This program thus represents a subject with broad connections across mathematics. Already, collaborations between experts coming from different origins have revealed deep connections between these subjects, but it seems clear that there is much more to come. This MSRI program will significantly expand the research connections between these fields, and lay the groundwork for a unified theory of quantum symmetry. In particular it seems unlikely without a semester program at the scale provided by MSRI that it will be possible to bring together all these groups, and hopefully synthesize new goals and directions for the study of quantum symmetries. Many of the subjects touched upon are young and growing, and the significant postdoctoral fellowship support provided as part of an MSRI program will ensure that leading new researchers will participate in, and help direct, the semester.

This is an exciting time for the field, with many important recent breakthroughs.  Some of these recent breakthroughs include:
\begin{itemize}
\item Using higher Frobenius--Schur indicators, Bruillard--Ng--Rowell--Wang resolved a major open problem by showing that there are only finitely many modular tensor categories of any given rank. 
\item Bartlett--Douglas--Schommer-Pries--Vicary gave a complete classification of 3,2,1-dimensional topological field theories in terms of modular tensor categories. 
\item Significant progress has been made in the past few years on the classification of small index subfactors, extending the prior classification to index 4 all the way up to 5.25 and finding several new examples. Some of these new examples are `quadratic categories', further studied by Izumi, and others appear truly exotic. These examples open the way to a rich new world of fusion categories which do not come from groups or quantum groups.  
\end{itemize}
Despite such progress, the field is still young and many important conjectures remain open. Some notable examples include:
\begin{itemize}
  \setlength{\itemsep}{1pt}
  \setlength{\parskip}{0pt}
  \setlength{\parsep}{0pt}
\item Can integral fusion categories be classified group theoretically?
\item Are fusion categories always pivotal? (An analogue of Kaplansky's 5th conjecture.)
\item Does the dimension of any simple object divide the global dimension? (An analogue of Kaplansky's 6th conjecture.)
\item What is the complete classification of quadratic fusion categories?
\item Is every modular tensor category realized as the representation theory of a conformal field theory?
\end{itemize}
Such a decisive time in the development of the field makes it the perfect time for an MSRI program.

We are interested in developing general theoretical approaches to quantum symmetry, but equally important is the search for new examples. Historically the main sources of examples are classical groups and specializations of the Drinfeld--Jimbo quantum groups which deform the universal enveloping algebras of Lie algebras. However, several key examples appearing from subfactor classification results do not appear to come from these worlds. These unusual examples can in turn inform the general theory, since they allow one to distinguish between results that may hold in general and ones that just hold for groups and quantum groups for reasons special to those situations.

Tensor categories have a planar structure, with the category structure going in one direction and the tensor structure in the other direction. This planar structure has played a key role in many contexts, from Jones's notion of planar algebras in subfactor theory, to the higher Frobenius--Schur indicators which appear as traces of rotation operators. More generally, there has been great recent interest in the broader subject of higher dimensional algebra, such as the study of $E_n$ algebras in $(\infty,1)$-categories. Tensor categories and braided tensor categories (which are the $E_1$ and $E_2$ algebras in the $(2,1)$-category of linear categories) give some of the most accessible examples of this general theory.

Finally, we want to particularly emphasize the rapid development of the study of quantum symmetries in condensed matter physics over the last decade. Physicists studying topological order have adopted the language of tensor categories, and their work suggests many new fundamental problems in the field. We intend that this semester program will include a substantial presence of physicists interested in topological order, and facilitate interactions between the mathematics and physics communities.

\section{Areas of concentration}
Within the framework of studying the various guises of quantum symmetries, and their interactions, we propose the following seven areas of focus for the semester.

\begin{enumerate}
  \setlength{\itemsep}{1pt}
  \setlength{\parskip}{0pt}
  \setlength{\parsep}{0pt}
\item Tensor categories, fusion categories, module categories, and applications to representation theory.
\item Braided, symmetric, and modular tensor categories.
\item Hopf algebras, their actions on rings, and classification of semisimple and of pointed Hopf algebras.
\item Subfactors, planar algebras, and analytic properties of quantum symmetries.
\item Quantum invariants of knots and 3-manifolds, and local topological field theories.
\item Conformal nets, vertex algebras, and their representation theories.
\item Topological order and topological quantum computation.
\end{enumerate}

We now briefly summarize the state of research in each area, highlighting connections and promising new directions.

\subsection{Tensor categories}
A tensor category is a linear abelian category with a unit object, a tensor product, and duality. Tensor categories can either be thought of as an abstraction of the representation theory of a group, or as a higher dimensional analogue of an algebra. Of special interest are the fusion categories, which have the same semisimplicity properties as the representation theory of a finite group over a field of relatively prime characteristic.  Each of these two points of view suggests different approaches to the study of tensor categories. For example, the analogy with groups has lead to definitions of solvability and nilpotence for fusion categories in the work of Etingof--Gelaki--Nikshych--Ostrik.  Similarly, the analogy with rings suggested the key role played by module categories and bimodule categories over tensor categories introduced by Bernstein, Crane, and I. Frenkel, and further developed by Ostrik and others.

Much of representation theory is the study of specific tensor categories. This includes classical examples, like the category of finite dimensional representations of a compact group, but also includes non-classical examples where the general theory of tensor categories is needed.  Some important examples of tensor categories appear in representation theory, for example, the fusion product of representations of affine Lie algebras (in the work by Kazhdan--Lusztig, Finkelberg and Huang), categories of representations of vertex operator algebras (Huang and Lepowsky), and categories of some perverse sheaves with tensor operation given by convolution (or, more generally, categories of Soergel bimodules) in the work of Lusztig, Soergel, Bezrukavnikov--Finkelberg--Ostrik, and Elias--Williamson. The development of a theory of tensor categories and fusion categories has lead to applications in representation theory, most notably in the celebrated work on the geometric Satake isomorphism by Lusztig, Ginzburg, Beilinson--Drinfeld, and Mirkovic--Vilonen. Some further applications include theory of Harish--Chandra bimodules (Bezrukavnikov--Finkelberg--Ostrik), finite W-algebras (Losev--Ostrik), and character sheaves (Bezrukavnikov--Finkelberg--Ostrik, Ben-Zvi--Nadler, and Lusztig).


One of the main current hurdles in the study of fusion categories is a lack of a robust extension theory. Such an extension theory would allow one to reduce the study of fusion categories to ``simple" ones along the same lines as the reduction of the study of all finite groups to the simple groups. Recently some interesting progress has been made in that direction, including Etingof--Nikshych--Ostrik's homotopy theoretical approach to graded extensions of fusion categories, and a notion of short exact sequence of tensor categories introduced by Brugui\`eres--Natale and generalized by Etingof--Gelaki. However, the critical example of Izumi's quadratic categories, which look as though they should be built by some kind of extension (see work of Evans--Gannon), do not fit into either of these contexts. Further development is needed.

There are many interesting classification questions concerning tensor categories. There is an appropriate notions of dimension for objects in a finite tensor category called Frobenius-Perron (FP) dimension, but the FP dimensions are algebraic integers and need not be rational integers.  The global FP dimension is the sum of the squares of the FP dimensions, and plays the role of the size of the group. So one natural question is to try to classify tensor categories of fixed global FP dimension.  In the special case where the global FP dimension is a rational integer, conjecturally all such fusion categories should be classifiable in terms of only group theoretical data.  This result is known when the global FP dimension has few prime factors, in work of Drinfeld, Etingof, Gelaki, Jordan, Larson, Nikshych, and Ostrik. In particular, an analogue of Burnside's $p^a q^b$ theorem was recently proved by Etingof--Nikshych--Ostrik. By contrast very little is known when global FP dimension is not a rational integer, or when the tensor category is not semisimple. Global FP dimension is not the only important measure of complexity for fusion categories: another natural measurement is the rank (which corresponds to the number of irreducible representations of a finite group), and a big open question is whether there are finitely many fusion categories of a given rank.


\subsection{Braided, modular, and symmetric tensor categories}
Just as  commutative and non-commutative algebra have quite different flavors, one should distinguish between the study of general tensor categories and ones where the tensor product is ``commutative." Here ``commutativity" means picking some isomorphisms $\sigma_{V,W}: V \otimes W \rightarrow W \otimes V$, but there's some flexibility in terms of what properties these commutors should have. The two natural choices are that they should satisfy the relations of the symmetric group or the relations of the braid group, and such tensor categories are called symmetric and braided respectively. Symmetric tensor categories typically are quite classical in nature, while braided tensor categories are more quantum and have played a critical role in the development of quantum knot polynomials and 3-manifold invariants. The most important braided fusion categories are the ones that are furthest from symmetric, and are called modular tensor categories.

Just as every ring has a center which is a commutative ring, every tensor category has a Drinfeld center which is a braided tensor category. Since having a braiding is an extra structure, rather than just a property like elements commuting in an algebra, the Drinfeld center is a lot richer than the original tensor category. Thus one of the main techniques throughout the study of tensor  categories is to take the Drinfeld center.  Examples include Etingof--Nikschych--Ostrik's homotopy theoretic approach to graded extensions of fusion categories, Ostrik's work on rank 2 and rank 3 fusion categories, and the result that dimensions of objects in fusion category are cyclotomic integers; all make use of the Drinfeld centre, even when stating results purely about the fusion category.

Modular tensor categories have many invariants, such as the $S$ and $T$ matrices, which play a similar role to the character table of a finite group. The $S$ and $T$ matrices give a projective representation of the modular group $\mathrm{SL}_2(\mathbb{Z})$.  Much as with character tables, the combinatorial properties of the $S$ and $T$ matrices are quite rich and restrictive. A huge amount of recent progress has been made by considering additional combinatorial invariants called generalized Frobenius--Schur indicators in work of Linchenko, Montgomery, Kashina, Sommerh\"auser, Zhu, Schauenberg, and Ng. These indicators are traces of certain rotation operators and allow for the definition of an analogue of the exponent of a group. Examples of recent deep results proved using this technique, are that the kernel of the $\mathrm{SL}_2(\mathbb{Z})$ representation is a congruence subgroup (Ng--Schauenberg), a Galois symmetry property of the S-matrix (Dong--Lin--Ng), and an analogue of Cauchy's theorem saying that if a prime ideal divides the global dimension then it must divide the exponent (Bruillard--Ng--Rowell--Wang). A spectacular application of these techniques was given by Bruillard, Ng, Rowell, and Wang, who proved that there are only finitely many modular tensor categories of a given rank.

The property $\mathsf F$ conjecture states that a braided fusion category is weakly integral (that is, its objects have dimensions which are square roots of integers) exactly if the representation of the braid group coming as the mapping class group action on the vector space associated to a punctured disc by the associated topological field theory has finite image. It is currently hard to tell whether this conjecture looks plausible only because of the relative scarcity of examples. In the other direction, it is tempting to believe that weakly integral categories are automatically weakly group-theoretical, and that braided weakly group-theoretical categories should have property $\mathsf F$.

There have also been some important recent developments in the theory of symmetric tensor categories. A theorem of Deligne's says that over the complex numbers and assuming certain natural growth conditions, all symmetric tensor categories are categories of representations of super groups. Nonetheless there are some very interesting examples which do not satisfy this growth condition. The representation theory of the classical families of Lie groups $GL_n$, $O_n$, and $SP_n$, turn out to behave uniformly in $n$. For example, the dimension of a representation of $GL_n$ whose highest weight is supported near one end of the $A_{n-1}$ Dynkin diagram has dimension given by a polynomial in $n$. Deligne and Milne realized that these patterns could be made more precise: there is a $1$-parameter family of symmetric tensor categories $\mathrm{Rep}(GL_t)$, such that when $t=n$ the semisimplification of this category recovers $\mathrm{Rep}(GL_n)$. When $t$ is generic, the category fails Deligne's growth condition, and the category looks like the ``large n limit" of the representation theory of $GL_n$.  Similar ideas were developed also by Cvitanovic and Vogel. Vogel and Patureau-Mirand gave interesting applications of these families to the study of Vassiliev invariants.  The HOMFLYPT and Kauffman skein theories can be thought of as quantum versions of $GL_t$ and $OSP_t$. Serganova--Entova-Aizenbud--Hinich have recently constructed the abelian envelope of $\mathrm{Rep}(GL_t)$ for integer $t$, confirming a conjecture of Deligne.

Deligne also developed a similar family $\mathrm{Rep}(S_t)$ which interpolates between the categories of representations of the symmetric groups. Much recent progress has been made in studying these categories by Etingof, Harman, Comes, Ostrik, and others. There is some tantalizing evidence, due to Deligne--Gross, Vogel, Cohen--de Man, and others, that suggests that there may be a similar $1$-parameter family of symmetric tensor categories which interpolates between the exceptional lie algebras. Finally, it is natural to wonder what happens in characteristic $p$. First, Ostrik has made some recent progress towards an analogue, described in characteristic $p$.  Second, a recent idea of Deligne's has led to characteristic $p$ analogues, described in work of Harman and Etingof-Harman-Ostrik. One interesting pheonomenon is that in these new families the parameter is a $p$-adic integer, rather than an element of the finite field.


\subsection{Hopf algebras}
A typical linear abelian category consists of certain $A$-modules for some algebra $A$, and so it is natural to wonder what kind of structure on $A$ will allow one to take tensor products and duals of $A$-modules. The most natural such structure is that of Hopf algebra, where the action of $A$ on $V \otimes W$ is implemented via a coassociative map $\Delta: A \rightarrow A \otimes A$. The classical examples of Hopf algebras are group rings, universal enveloping algebras, and rings of polynomial functions on algebraic groups.  Non-classical examples include Taft algebras and Drinfeld--Jimbo quantized universal enveloping algebras.  The study of Hopf algebras predates the study of tensor categories by several decades (in work of Montgomery, Nichols, Radford, and Sweedler), but nonetheless blossomed after the discovery of quantum groups. By an analogue of Tannakian reconstruction, Hopf algebras are essentially the same thing as tensor categories together with a forgetful functor to vector spaces, so they can be understood using the general theory of tensor categories. 

A lot of early progress in Hopf algebras was driven by Kaplansky's 10 conjectures. Although most of these have been settled, a few important questions remain open. Kaplansky's 5th conjecture, that the antipode is an involution when $A$ is semisimple, remains open when the characteristic is $p>0$ and simultaneously $A^*$ is not semisimple. Kaplanasky's sixth conjecture that the dimension of any simple $A$-module divides the dimension of $A$ is still quite open. The best result is that the dimension of any simple module for the Drinfeld double (the Hopf algebra counterpart of the Drinfeld center) divides the dimension of $A$. There has also been a lot of interesting work in classifying small dimensional Hopf algebras, typically focused on either the semisimple case (see work of Natale and others) or in the pointed case (see work of Andruskiewitsch, Heckenberger, Schneider, and others). In the semisimple case, one can speculate that there could be a purely group theoretical classification in any finite dimension. In the pointed case, the Hopf algebras behave somewhat like quantized universal enveloping algebras, and much of the classification goes through Nichols algebras. After the work of Heckenberger there has been rapid progress in classification of Nichols algebras. 

Finally, there is a natural notion of an action of a Hopf algebra on a ring, and so Hopf algebras can be thought of as ``quantum symmetry groups" of rings. Much progress has been made in understanding which rings admit quantum symmetries and which do not (see work of Chan, Etingof, Kirkman, Kuzmanovich, Walton, Wang, Zhang, and others).  For example, Etingof--Walton showed that any action of a semisimple cosemisimple Hopf algebra on a commutative algebra factors through a group action, and in joint work with Cuadra they showed that the action of any finite dimensional Hopf algebra on a Weyl algebra factors through a group action. The latter of these ``no quantum symmetries" results was proved via reduction to characteristic $p$. By contrast, commutative algebras can admit non-semisimple quantum symmetries.

\subsection{Subfactors and planar algebras}
To study a Galois field extension $K \subset M$, we begin by looking at the
Galois group of automorphisms of $M$ fixing $K$. For example, the intermediate
fields $K \subset L \subset M$ are parameterized by the subgroups of the
Galois group. This theory finds a noncommutative, or quantum, analogue in the
study of operator algebras, as the \emph{standard invariants of subfactors}.

The subject began with the striking observation by Jones that the index of a
subfactor is \emph{quantized}, and the possible values of the index are
exactly $\{4 \cos^2(\pi/n)\}_{n \geq 3} \cup [4,\infty]$. We now understand
that these restrictions are the consequence of a rich and rigid algebraic
structure underlying any subfactor.
Subfactors provide an alternative axiomatization of the notion of quantum
symmetry, especially adapted to the unitary setting. A subfactor is an
inclusion of von Neumann algebras with trivial centres. From a subfactor, one
can extract a finite algebraic gadget called the \emph{standard invariant}.
There have been many alternative ways to axiomatize and study these objects
(for example as Ocneanu's paragroups, Popa's $\lambda$-lattices, and Jones'
planar algebras). For our purposes, a particularly interesting
alternative is as a pair $(\cC, A)$, where $\cC$ is a unitary pivotal fusion
category, and $A$ is a $C^*$ algebra object in $\cC$. The importance of subfactors
in the subject of quantum symmetries is seen in the fact that every such pair
in fact comes from some subfactor!

The techniques developed in the course of analyzing the standard invariants of
subfactors have resulted in unusual constructions of examples as well as
powerful obstructions to the existence of certain families. Indeed, the most
exotic known fusion categories at present --- the categories of $A-A$ or $B-B$
bimodules, for $A \subset B$ the extended Haagerup subfactor ---  have been
constructed as part of the program of classifying small index subfactors, and are apparently
unobtainable starting from finite groups or quantum groups.
The various alternative axiomatizations of a standard invariant emphasize
different aspects of the algebraic structure, and typically suggest rather
different approaches1. It seems clear that
these alternative perspectives have not been exhausted, and indeed will have
impact throughout the study of quantum symmetries. 

Recent highlights of research into subfactors with consequences for the broader understanding of quantum symmetries include:
\begin{itemize}
  \setlength{\itemsep}{1pt}
  \setlength{\parskip}{0pt}
  \setlength{\parsep}{0pt}
\item The classification of subfactors to progressively higher and higher indices (in work Haagerup, Haagerup--Asaeda, Jones, Bigelow--Morrison--Snyder--Peters, Morrison-Snyder, Afzaly--Morrison--Penneys, Liu, and many others). In particular, these classifications have produced unexpected and very poorly understood examples, and further, provided unexpected evidence that high `supertransitivity' is very rare (tantalizingly analogous to the fact that highly transitive group actions are rare).
\item Amongst the examples discovered during the classification program have been several fusion categories now considered as part of the family of near group fusion categories, or more generally quadratic fusion categories. Quite a lot is now known, especially by work of Izumi and Evans--Gannon, and many more examples have been constructed. The complete theory has not been achieved, but may be close.

\item In addition to algebraic aspects of subfactor theory, there are deep analytic questions about subfactors. Just as geometric group theorists study analytic properties of groups such as amenability, Kazhdan's property ($\mathsf{T}$), or the Haagerup property, one can also use the theory of subfactors to generalize these properties to quantum symmetries. Amenability has played a key role for a long time, since Popa proved that amenable standard invariants can be realized essentially uniquely as subfactors of the hyperfinite $\mathsf{II}_1$ factor.
Recent exciting work of Popa--Vaes, Neshveyev--Yamashita, and Ghosh--C. Jones on the analytic representation theory of standard
invariants and rigid C*-tensor categories gives a systematic treatment
for these approximation and rigidity properties.

Although the standard invariants of non-amenable subfactors fall outside the world of finite quantum symmetries (having in particular infinitely many non-isomorphic simple bimodules), they constitute an intriguing adjacent neighborhood. Very little is known about the realization of the Templerley--Lieb--Jones standard invariant at a generic index, as a subfactor  of the hyperfinite $\mathsf{II}_1$ factor. 

\item Every fusion category sits inside its \emph{Brauer--Picard groupoid} of Morita equivalent fusion categories. This groupoid is particularly interesting for studying extension theory. The Brauer--Picard groupoids for several of the fusion categories first discovered as categories of modules of an exotic subfactor have recently been studied by Grossman--Snyder, mostly via combinatorial techniques. In one case, this study revealed that an apparently isolated example was in fact related to the family of quadratic categories.
\item Intermediate subfactors $N \subset P \subset M$, and more generally lattices of inclusions of factors, have also been studied intensely in recent years, beginning with the work of Bisch, Bisch--Jones, and Bisch--Haagerup. Recent results include Xu's classification of intermediate subfactors of Goodman--de la Harpe--Jones subfactors, Grossman--Snyder's work on intermediate subfactors coming from the Haagerup subfactor, and Liu's classification of subfactors of index $3+\sqrt{5}$ that have an intermediate subfactor of index $2$.
\end{itemize}


\subsection{Quantum invariants and topological field theory}
The connection between quantum invariants of knots and quantum symmetries goes right back to the beginnings of both subjects. Jones' discovery of his knot polynomial arose through studying the structure of the standard invariant of a subfactor. He realized that the Temperley--Lieb algebras are always present inside the tower of relative commutants of a subfactor (one of the earliest views of the standard invariant of a subfactor), and subsequently realized that there are homomorphisms from the braid groups into the Temperley--Lieb algebras which can be used to build the Jones polynomial. Since then, the Jones polynomial has been massively generalized, and we now understand that any object in a braided pivotal category gives rise to a knot invariant --- the original case of the Jones polynomial comes from the standard representation of $U_q \mathfrak{sl}_2$.

%These knot invariants were adapted by Witten and Reshetkhin--Turaev to give invariants of $3$-manifolds.  Furthermore, these invariants of $3$-manifolds have the additional structure of a topological quantum field theory. As axiomatized by Atiyah and Segal, an $n$-dimensional topological quantum field theory is a symmetric tensor functor from a bordism category to the category of vector spaces, which assigns a vector space to each closed $(n-1)$-manifold and a linear map to each $n$-dimensional bordism.

Moreover, beginning with the work of Witten, Reshetikhin--Turaev, and many others, leading up to the recent work of Lurie, we understand a deep relationship between higher categories and topological field theories, now expressed via the Baez--Dolan cobordism hypothesis. Higher categories with appropriate invariance conditions give rise to local topological field theories. Local topological field theories in $(d+1)$-dimensions assign algebraic data to all manifolds of degree at most $d$, and the algebraic data associated to a point recovers a higher category, which we should think of as the quantum symmetries of the field theory. The cobordism hypothesis explains in detail how these two constructions give a correspondence. The broadest definition of a quantum symmetry should be exactly those higher categories that appear in some version of this correspondence.

Although topological field theory was initially thought of as a way to apply algebra to topology, much of the interesting recent research goes the other way: using topology to organize complicated algebra. Examples include the work of Ben-Zvi and Nadler on topological field theory and character sheaves, work of Ben-Zvi--Brochier--Jordan which realize several rings of representation theoretic interest as coming from the value of certain surfaces in a topological field theory, and work of Douglas--Schommer-Pries--Snyder showing that Radford's theorem on the quadruple dual is exactly the belt trick in topology.

Another important direction is giving more explicit and geometric descriptions of the topological field theories guaranteed by the cobordism hypothesis. Factorization homology, particularly as recently developed in work of Ayala--Francis, has been quite successful in this direction giving a new, highly geometric, proof of the cobordism hypothesis.

There has been a lot of recent progress specifically in $3$-dimensions. In the fully local case, Douglas--Schommer-Pries--Snyder show that for TFTs with values in tensor categories, the values at a point (that is, the possible quantum symmetries) are precisely the fusion categories with nonzero global dimension. For 321 extended TFTs (which assign categories to closed 1-manifolds but nothing to points) recent work of Bartlett--Douglas--Schommer-Pries--Vicary provides a complete classification in terms of slight variations on modular tensor categories.  Finally, even in the non-extended setting a classification result was recently proved by Juhasz.

There is still a great deal of work to be done understanding what the cobordism hypothesis tells us in each dimension, and in other contexts. It seems likely that answers to these questions will give extremely profound indications of the most important classes of quantum symmetries, and will direct much of future research in the field.

In another direction, link homology theories (both Khovanov homology and Heegaard Floer homology) show signs that they are controlled by 4-categorical structures, categorifying the braided tensor categories of representations of quantum (super)groups. The resulting link and manifold invariants satisfy gluing formulas analogous to those in TFT invariants. 

\subsection{Conformal field theory}
The connection between conformal field theory, and quantum symmetry described in the previous sections, is far from well understood. Nevertheless, an essential connection does exist, and there are many tantalizing hints of its importance.

Certainly, the representation category of a rational conformal field theory gives a modular tensor category. It is possible, and indeed strongly suspected by some, that in fact every modular tensor category is realized in this way. Thus, as is the case with subfactors, conformal field theory can be thought of as a host (perhaps universal) for certain types of quantum symmetries. Many early examples of quantum symmetries were first constructed via conformal field theory, although in many cases subsequently a purely algebraic approach has been developed --- often enough transporting some construction earlier known in CFT to the categorical setting. It is very likely that CFT constructions will inspire further development of structure theory for fusion categories and modular tensor categories.

There are alternative axiomatizations of conformal field theories, e.g.  as vertex operator algebras or as conformal nets. The precise relationship between these definitions is still being worked out, but conformal nets are manifestly unitary, while the vertex algebra approach includes non-unitary examples. The conformal net approach is intimately related with the theory of subfactors, and this connection has been developed extensively by Kawahigashi, Longo, and collaborators.

In another direction, the correspondence set up by the cobordism hypothesis between local field theories and higher categorical data works with any ambient target category. In particular, field theories valued in conformal nets seem particularly interesting, and are being studied by Bartels--Douglas--Henriques, and others.

One of the most important examples of a conformal field theory is the one whose classical symmetries are the Monster group, as developed by Frenkel--Lepowsky--Meurman and used to prove the Moonshine conjectures by Borcherds. Conformal field theory seems to be the natural home for the Monster group and several other sporadic simple groups, and so perhaps it is reasonable to think of those groups as classical  shadows of a something more naturally living in the world of quantum symmetry.

\subsection{Topological order}
The discovery of topological order in condensed matter physics is a
fundamental theoretical breakthrough --- and hopefully in the future, with
some engineering breakthroughs, the foundation of new technologies including
topological quantum computation. Quantum symmetries promise to provide the
mathematical foundations for this physics and engineering.

Topological order goes beyond the standard theory of symmetry breaking, going
back to Landau, and can be understood in terms of systems whose `symmetries'
are not classical symmetries described by groups, but rather quantum
symmetries as discussed here. In particular, the fascinating realization is
that there are certain physical systems with effective descriptions via local
topological field theories. For physicists, topological order can be
characterized by the presence of robust ground state degeneracies, and these
degeneracies are exactly the vector space assigned by the governing
$d+1$-dimensional topological field theory to the $d$-manifold describing the
shape of the system. The higher categorical data determining the local
topological field theory should be thought of as the quantum symmetry of the
topological phase. In particular, for $2+1$-dimensional topological field
theory, this data is precisely a fusion category, as discussed above.

A simple form of topological order is the class \emph{symmetry protected phases}, in which there is a finite group symmetry $G$, without which the system would become trivial. These have been identified with the Dijkgraaf--Witten TFTs, associated to the pointed fusion categories $\mathsf{Vec}_\omega^G$, that is, $G$-graded vector spaces with associator twisted by a 3-cocycle $\omega$. These are currently under intensive investigation by physicists (including Burnell, Chen, Gu, Levin, Vishwanath, Wen, and others). More complicated topological orders have tentatively been observed in fractional quantum Hall effect systems, but much remains to be unravelled here.

One of the most exciting advances, from a mathematical perspective, has been the discovery of Levin--Wen string net models. These provide a \emph{local Hamiltonian realization} of any $(2+1)$-dimensional local TFT. They have been generalized recently by the Walker--Wang models in $(3+1)$-dimensions, and in fact their construction can be seen as a general recipe for constructing local Hamiliton models for local topological field theories, with the terms of the Hamiltonian directly constructed from the gluing formulas for a handle decomposition of the underlying manifold.

The outstanding prospective application of topological order, and the mathematical theory of quantum symmetry underlying it, is in topological quantum computing, as pioneered by Freedman, Kitaev, and others. Topological quantum computing offers an intrinsically robust underlying hardware model for quantum computation, in which the quantum information is encoded in a topologically protected degenerate ground state. Through the local Hamiltonian realizations of local TFTs, it is now a possibility that even in a hardware model for quantum computation that is not topological, one could consider first simulating a topological phase, and encoding the quantum computation inside that simulation. Indeed, the topological phase associated to the fusion category $\textsf{Vec}_{\mathbb Z / 2 \mathbb Z}$ was originally named the \emph{toric code} by Kitaev for just this reason.

We note that several prominent members of the topological quantum computation community   have indicated their interest in participating in this semester (e.g.  Kuperberg, Walker, and Wang on the mathematical side; we're hoping to invite more from the physics side as well).

\section{Goals at MSRI}
We hope that this semester will energize the field in a manner only possible at MSRI. With quantum symmetries being studied from so many points of view, with a variety of different tools and objectives, it is an ideal time to gather representations of all the contributing fields. There's a certain amount of catching up to do --- various tools and techniques have been developed very far in narrow directions, but almost certainly have wider application across the study of quantum symmetries.  At the same time, we hope that the semester will result in the synthesis of overarching goals for the field, and enable a wider perspective on the entire subject than has hereto been possible. A semester program at MSRI is at sufficient scale that it is possible to bring together all these groups.

We intend to structure the bulk of the semester around working groups and learning seminars. As we learn when various representatives will be present, we will try to coordinate groups that want to tackle particular problems, and encourage clusters of active new research collaborations. At the same time, particularly because there will be researchers with quite different backgrounds present, we will encourage sequences of learning seminars aimed at explaining the core ideas and methods of one sub-field, to participants in other sub-fields. We particularly hope that the MSRI postdoctoral fellows will take the lead in directing these seminars.

The semester at MSRI will to the extent possible involve researchers at nearby institutions (for example Borcherds, Reshetikhin, Serganova, Teichner, Teleman, to the extent they are present, at Berkeley, Kuperberg and Vazirani at Davis, and the Microsoft Station Q group at Santa Barbara), in particular encouraging their students and postdocs to be involved in the activities at MSRI.


\section{Related Programs}

We list below a few recent programs that overlapped with one of our streams, but note that none of them have overlapped with more than one of our 7 streams.

\paragraph{Programs:}

\begin{description}
  \setlength{\itemsep}{1pt}
  \setlength{\parskip}{0pt}
  \setlength{\parsep}{0pt}
\item[Subfactors and their applications \href{https://www.newton.ac.uk/event/oas}{program}] (Isaac Newton Institute, early 2017) \\
The program on subfactors at INI will overlap substantially with concentration 4 of the current proposal, but only incidentally with
the other aspects. In particular, the participants there are almost exclusively operator algebraists, as opposed to our program which will draw heavily from the representation theory, topology, and quantum computation communities.  
\item[Von Neumann algebras \href{https://www.him.uni-bonn.de/programs/future-programs/future-trimester-programs/von-neumann-algebras-2016/description/}{trimester}] (Hausdorff Institute for Mathematics, early 2016) \\
The trimester program on von Neumann algebras at the Hausdorff Institute in 2016 is substantially disjoint from
the proposed program; although many of the subfactor community will be present, the primary focuses of the program are deformation 
rigidity issues, and free probability.
\item[Homotopy theory, fields, and manifolds \href{https://www.him.uni-bonn.de/programs/past-programs/past-trimester-programs/homotopy-theory-2015/description/}{trimester}] (Hausdorff Institute for Mathematics, early 2015) \\
The past program, `Homotopy theory, fields, and manifolds' had a substantial homotopic theoretic
focus (cf. the Felix Klein lectures by Rezk) which was unrelated to the current proposal. It had some overlaps with concentration 5 of the current proposal, particularly through the factorization homology approach to higher dimensional topological field theories. On the other hand, 
our proposal focuses particularly 
on low dimensional field theories, and emphasizes the interaction with the study of tensor categories in particular.
\end{description}

There have not been any recent substantially related programs at MSRI since the 2000-2001 program on operator algebras studied subfactors and their standard invariants. Of course, the famous parallel programs in low-dimensional topology and in operators algebras during 1984-1985 led directly to much of the fantastic interplay between topology and quantum symmetry described in this proposal!
% cf Vaughan's remarks in https://math.berkeley.edu/~vfr/msri.pdf

There have been several shorter workshops at institutes which are more directly related to the topic (many of which share organizers with this proposal).  These illustrate the recent excitement in the field, but cannot substitute for the more substantial scope of MSRI semester.

\paragraph{Conferences:}
\begin{itemize}
  \setlength{\itemsep}{1pt}
  \setlength{\parskip}{0pt}
  \setlength{\parsep}{0pt}
\item Fusion categories and subfactors (Banff, October 2018)
\item Modular categories (Oaxaca, Mexico, August 2016)
\item Nichols algebras and their interactions with Lie theory, Hopf algebras and tensor categories (Banff, September 2015)
\item Factorizable structures in algebra and geometry (Banff, August 2015)
%\item Subfactor Theory in Mathematics and Physics (Qinhuangdao, China, July 2015)
\item The mathematics of conformal field theory (Australian National University, July 2015)
\item Subfactors and Conformal Field Theory (Oberwolfach, March 2015)
\item Symmetry and Topology in Quantum Matter (IPAM, January 2015)
\item Subfactors and fusion categories (Banff, April 2014)
%\item Fusion categories (Dijon, May 2013)
\item Classifying fusion categories (AIM, March 2012)
%\item Subfactors in Maui (Hawaii, July 2011/2012/2013/2014)
\end{itemize}

\section{Companion program}
We are very excited about the proposed companion program on higher category theory.
Many of the topics in our program can be thought of as low-dimensional examples of higher categorical structures.  For example, tensor categories and braided tensor categories are $E_1$ and $E_2$ algebras in the $2$-category of categories, and the collection of all tensor categories forms a $3$-category.

The two programs will go very well together. We anticipate that for those researchers working across both areas, the simultaneous programs will be irresistible. Moreover, with some coordination between the organizers of the two programs, we can ensure that the workshops for each program are of considerable interest to the participants of the other program. We will make sure that each workshop has at least one talk, scheduled early, that outlines the connections and necessary background for participants in the other program.


\section{Key participants}
We have removed the list of potential key participants from this document, as the attached Excel spreadsheet contains this information.

For the updated proposal submitted December 2016, we have now asked some of the senior participants about the time they will be able to commit to attending the program; their responses appear here.
\begin{description}
  \setlength{\itemsep}{1pt}
  \setlength{\parskip}{0pt}
  \setlength{\parsep}{0pt}
\item[David Ben-Zvi] I'd like to come for the whole semester if I can get off teaching.
\item[Dietmar Bisch] I would definitely like to come to the program for
at least a month. If there is a possibility of financial support from MSRI,
it looks likely that I would come for the whole semester.
\item[Inna Entova] Up to two weeks, maybe twice.
\item[Andr\'e Henriques] I will definitely try to come (for as long as my family circumstances allow it).
\item[Masaki Izumi] I'm very much interested in paying an extended visit to MSRI for Quantum Symmetries. 
\item[Vaughan Jones] Put me down for a month. It could conceivably work out that I'll be there for
a semester but that's not the current long term plan.
\item[David Jordan] I was also contacted by the organizers of the parallel conference. I am extremely excited that these two workshops are happening, and I will definitely plan to come for the whole semester.
\item[Yasuyuki Kawahigashi] I certainly hope to to be at MSRI for an extended period, more than a month.
\item[Scott Morrison] I plan to attend the entire semester; I have made arrangements for a sabbatical.
\item[Dmitri Nikshych] I doubt that my teaching and family duties will allow me to come for
a month or more, but I could certainly try to make several weekly 
visits. Good luck with the proposal.
\item[Victor Ostrik] My intention is to take sabbatical for the time of the program,
so I hope that I will come for the whole term.
\item[David Penneys] I'll come for a month.
\item[Julia Pevtsova] Yes, I'd be interested in coming for a period of time; possibly for a semester.
\item[Eric Rowell] I am sure I can arrange to be there for at least a month, and with sufficient notice I could do my sabbatical that academic year.
\item[Claudia Scheimbauer] I would be very happy to attend for an extended time period, or even the whole semester. I was also asked to give a response to the potential companion program on higher categories and the combination of the two programs would be ideal for my research interests!
\item[Peter Teichner] I am planning on being there for the entire semester, I’m also involved with the sister application in Higher Categories. I very much hope that both programs will be approved because I am convinced that they could profit from each other.
\item[Chelsea Walton] I would be interested in coming for the whole semester.
\end{description}

We also include here some comments received in reply to our initial email (October 2015). There are many more included in the personnel spreadsheet.
\begin{description}
  \setlength{\itemsep}{1pt}
  \setlength{\parskip}{0pt}
  \setlength{\parsep}{0pt}
\item[David Ben-Zvi] I guess I missed your deadline, but sure I'm very enthusiastic! and feel free to use my name on future revisions.
\item[Alexandru Chirvasitua] Word of this proposed MSRI semester you are putting together reached me through acquaintances, and I noticed the original message asked for shows of support. I just wanted to drop a word and say I would be very interested in attending, if it does come to fruition!
\item[Andre Henriques] Lots of enthusiasm! Yes! Good! Hurray! Go for it!
\item[Masaki Izumi] Of course I'm interested in such a program.
\item[Yasuyuki Kawahigashi] Thank you very much for your message. I would like to attend the program very much. I really hope the application will be successful!
\item[Alexander Kirillov] Yes, I would be interested in attending the program. Please keep me posted!
\item[Greg Kuperberg] It sounds great to me!   MSRI is driving distance from Davis, so it's very convenient.  And I'm definitely interested in the list of topics, and so are many of my students.   I don't know exactly what students I will have then, but odds are they will be interested.
\item[Zhengwei Liu] Sounds great! All the subjects seem interesting.
\item[Michael M\"uger] This is an excellent initiative, and I certainly support it. Feel free to put me on your list. 
\item[Sonia Natale] A program at MSRI sounds great, it would be a pleasure to participate.
\item[Nicolai Reshetikhin] This sounds fantastic. I will make sure to be in Berkeley with light teaching load and will certainly participate. The program will be very timely. I am sure there will be many new stuff in the area by that time. Quantum groups at roots of unity are unfolding.
\item[Claudia Scheimbauer] I would absolutely be interested in an MSRI program on these topics.
\item[Hans-J\"urgen Schneider] This is a great idea, a semester program on quantum symmetries in a broad sense. My main interest is in Hopf algebras and quantum quantum groups, and in tensor categories. I would be very much interested in such a program, to exchange ideas, and in particular to learn about related topics.
\item[Peter Teichner] Thanks for the invitation, I'm definitely interested in participating, maybe even for the entire semester.
\item[Chelsea Walton] I am very much enthusiastic about this idea! I hope the proposal gets accepted! Let me know if you need anything.
\item[Pinhas Grossman] Thanks for the information about the proposed semester. It sounds like a very exciting program, bringing together a diverse collection of research areas with strong thematic connections. I would certainly be very keen to participate in such a program.
\item[Kevin Walker] You can add me to the list of people who are interested.
\item[Zhenghan Wang] Indeed I am very enthusiastic for such a program, and interested in the participation in the program.
\item[Sarah Witherspoon] This sounds great! I am certainly interested in hearing more about this as time goes on.
\end{description}

As the planning horizon is still quite advanced, it is impossible to know who will be the best candidates for postdoctoral fellows during the program. In the spreadsheet for this proposal, we have marked the current graduate students and postdocs, to give an indication of the relevant numbers.

Nevertheless, as the postdoctoral fellows form an extremely important part of the program, we will remain in close contact with all of the key participants of the program, asking them to advertise the possibility of postdoctoral positions, and to alert us to promising candidates. As many fields involved in the proposal are new and fast moving, we hope that junior researchers will play an active role in the program, including as speakers during all three of the workshops. We will coordinate with the postdoctoral fellows to organize regular seminar series at MSRI, both for the postdocs to talk about their own research, and to ensure close cooperation between the postdocs, short term visitors, and long term visitors. As the event comes closer and we have indications of the times that various participants will be present, we will ensure that special sequences of MSRI seminars coincide with particular periods that sub-fields are strongly represented.

We also note that we are organizing a summer school for graduate students on subfactors, planar algebras, and tensor categories at MSRI in 2017. We hope that this school will help build enthusiasm for these relatively new subjects, and at the same time will give us an excellent opportunity to identify potential postdoctoral participants at the MSRI program --- the timing here could not be better!

\section{Human Resources}
The organizers are committed to encouraging the participation of members of groups which are underrepresented in mathematics. Noah Snyder and Emily Peters will spearhead this effort. On the spreadsheet of potential participants we have indicated women and underrepresented minorities. We plan to contact these people after approval so that they can make plans to attend, and also to ask them to suggest other participants in order to tap into their professional networks. For example, Chelsea Walton (a likely participant, and potential co-organizer for the Connections for Women workshop) maintains an updated list of women in non-commutative algebra and representation theory at \url{www.women-in-ncalg-repthy.org}, which will be especially useful when it comes time to find postdocs. We will also ask senior participants to suggest potential women postdocs. The 2017 MSRI summer school for graduate students will be an excellent opportunity to identify further candidates. We will advertise the postdoc positions in the Association for Women in Mathematics (AWM) newsletter.  We are committed to maintaining an environment safe from harassment, and our conferences will have a code of conduct that makes it clear that harassing behavior will not be tolerated.  We will also advertise the postdoctoral positions via the National Association of Mathematicians (NAM) and the Society for Advancement of Chicanos/Hispanics and Native Americans in Science (SACNAS).

We are submitting along with the proposal a detailed spreadsheet containing the personal data (names, affiliations, country, contact emails, gender, minority status) of potential participants. This spreadsheet also includes snippets of potential participants replies to our initial proposal --- it's easy to see from these how much enthusiasm there is for this proposal!


\section{Workshops}
The program will begin with a Connections for Women workshop, with Emily Peters and Chelsea Walton as the lead organizers; other possible organizers include Claudia Scheimbauer, Fiona Burnell, and Monica Vazirani (who has past experience organizing CfW programs). We anticipate coordinating with the introductory conference on invitations and speaker lists.  We hope that the Connections for Women workshop will serve to highlight recent work by women in the field. At the same time, it will be an opportunity for attendees to extend their mathematical contacts, and to discuss professional issues particularly relevant to female researchers in the field and in mathematics generally.  The organizers envision three to four hour-long talks per day, with speakers instructed to focus on making the material accessible to graduate students and researchers in different fields.  There will be many short breaks so that participants can talk to each other, as well as a question and answer session, and social activities.

For the introductory workshop, we suggest having one speaker for each of the 7 streams of the conference, each giving two talks.  The speakers will be chosen as the best expository speakers, able to give clear and motivated introductions to a broad audience. We will ask the speakers to talk about the background of their field as well as to give suggestions for open problems and future directions, to set the stage for the semester's work. Possible speakers include
\begin{itemize}
  \setlength{\itemsep}{1pt}
  \setlength{\parskip}{0pt}
  \setlength{\parsep}{0pt}
\item Victor Ostrik (fusion categories) 
\item Eric Rowell (modular tensor categories)
\item Peter Schauenburg (Hopf algebras)
\item Emily Peters (subfactors and planar algebras) % alternatives: Dave Penneys, 
\item Chris Schommer-Pries (local field theories) % alternatives: Noah, Chris D, Teichner
\item Andre Henriques (conformal nets) % alternatives: Terry Gannon, Chris Douglas
\item Zhenghan Wang (topological order and quantum computing) % alternatives: Eric Rowell, physicists?
\end{itemize}
and closer to the date appropriate junior researchers will be invited. 
We will strongly emphasize to each of the speakers the importance of giving accessible and broadly interesting talks!
Vaughan Jones, Emily Peters, Noah Snyder, and Victor Ostrik will lead the organization of this workshop.

We propose that the topical workshop be on \emph{Tensor categories and topological quantum field theories}. These two subjects are quite closely intertwined, and it was the relationship between tensor categories and Turaev--Viro and Witten--Reshetikhin--Turaev topological field theories which drove much of the study of tensor categories in its early days.  Recently, the development of extended topological field theories has revitalized these two fields and made the connection between them more explicit. In one direction, topological field theory gives topological approaches to studying tensor categories. In the other direction, tensor categories give some of the most accessible examples of topological field theories.  We plan to have four talks per day, with no talks on Wednesday afternoon. It is important that the conference allows time for informal discussions and collaboration.

Important recent developments and current hot areas of research in this topic include:
\begin{itemize}
  \setlength{\itemsep}{1pt}
  \setlength{\parskip}{0pt}
  \setlength{\parsep}{0pt}
  \item The cobordism hypothesis and the classification of fully local topological field theories.
\item Factorization homology, and topological field theories attached to $E_n$ algebras.
\item Applications of the cobordism hypothesis to tensor categories and $3$-dimensional topological field theories.
\item Classification of extended, but not fully-local, $321$-dimensional field theories.
\item DAHA-valued knot invariants.
\item Higher categorical foundations of topological field theory.
\item Khovanov homology as a $4$-dimensional topological field theory.
\item Partially defined topological field theories, and non-semisimple topological invariants.
\end{itemize}



Potential organizers would be John Francis, David Jordan, Scott Morrison, Eric Rowell, Claudia Scheimbauer, and Chris Schommer-Pries.
Potential speakers would include:
\begin{multicols}{2}
\begin{itemize}
  \setlength{\itemsep}{1pt}
  \setlength{\parskip}{0pt}
  \setlength{\parsep}{0pt}
\item David Ayala	(Montana State)
\item David Ben-Zvi (UT Austin)
\item Chris Douglas	(Oxford)
\item John Francis (Northwestern)
\item Dan Freed	(UT Austin
\item David Jordan (Edinburgh)
\item Anton Kapustin	(Caltech)
\item Alexander Kirillov	(Stony Brook)
\item Claudia Scheimbauer	(MPIM Bonn)
\item Chris Schommer-Pries	(MPIM Bonn)
\item Stephan Stolz	(Notre Dame)
\item Peter Teichner	(MPIM/Berkeley)
\item Theo Johnson-Freed	(Northwestern)
\item Vladimir Turaev	(Indiana)
\item Kevin Walker (Microsoft)
\item Jacob Lurie (Harvard)
\end{itemize}
\end{multicols}

\section{Keywords}
Suggested keywords:
\emph{
quantum symmetries, monoidal categories, braided categories, modular tensor categories, fusion categories, planar algebras, subfactors, conformal field theory, conformal nets, topological phases of matter, topological quantum computing, topological field theory, cobordism hypothesis, quantum groups, Hopf algebras, knot polynomials}

Suggested MSC codes:
\begin{itemize}
  \setlength{\itemsep}{1pt}
  \setlength{\parskip}{0pt}
  \setlength{\parsep}{0pt}
\item[16T05] Hopf algebras and their applications
\item[16T20] Ring-theoretic aspects of quantum groups
\item[17B37] Quantum groups (quantized enveloping algebras) and related deformations
\item[17B68] Virasoro and related algebras
\item[17B69] Vertex operators; vertex operator algebras and related structures
\item[18D05] Double categories, $2$-categories, bicategories and generalizations
\item[18D10] Monoidal categories (= multiplicative categories), symmetric monoidal categories, braided categories
\item[18D35] Structured objects in a category (group objects, etc.)
\item[20F36] Braid groups; Artin groups
\item[20G42] Quantum groups (quantized function algebras) and their representations
\item[46L37] Subfactors and their classification
\item[46L60] Applications of selfadjoint operator algebras to physics
\item[57M27] Invariants of knots and 3-manifolds
\item[57R56] Topological quantum field theories
\item[58B32] Geometry of quantum groups
\item[81P45] Quantum information, communication, networks
\item[81P68] Quantum computation
\item[81R10] Infinite-dimensional groups and algebras motivated by physics, including Virasoro, Kac-Moody, $W$-algebras and other current algebras and their representations
\item[81R15] Operator algebra methods
\item[81R50] Quantum groups and related algebraic methods
\item[81T05] Axiomatic quantum field theory; operator algebras
\item[81T40] Two-dimensional field theories, conformal field theories, etc.
\item[81V70] Many-body theory; quantum Hall effect
\end{itemize}


\section{One page summary}
Symmetry, as formalized by group theory, is ubiquitous across mathematics and science. Classical examples include point groups in crystallography, Noether's theorem relating differentiable symmetries and conserved quantities, and the classification of fundamental particles according to irreducible representations of the Poincar\'e group and the internal symmetry groups of the standard model. However, in some quantum settings, the notion of a group is no longer enough to capture all symmetries.  Important motivating examples include Galois-like symmetries of von Neumann algebras, anyonic particles in condensed matter physics, and deformations of universal enveloping algebras. The language of tensor categories provides a unified framework to discuss these notions of quantum symmetry.

Within the framework of studying the various guises of quantum symmetries, and their interactions, we will focus on following seven areas.
\begin{enumerate}
  \setlength{\itemsep}{1pt}
  \setlength{\parskip}{0pt}
  \setlength{\parsep}{0pt}
\item Tensor categories, fusion categories, module categories, and applications to representation theory.
\item Braided, symmetric, and modular tensor categories.
\item Hopf algebras, their actions on rings, and classification of semisimple and of pointed Hopf algebras.
\item Subfactors, planar algebras, and analytic properties of quantum symmetries.
\item Quantum invariants of knots and 3-manifolds, and local topological field theories.
\item Conformal nets, vertex algebras, and their representation theories.
\item Topological order and topological quantum computation.
\end{enumerate}

Potential graphics to use alongside the one-page summary:
\begin{itemize}
\item An associahedron: $$\mathfig{0.4}{A4-vertices}$$
\item An equation from the skein theory of quantum $G_2$: $$\mathfig{0.75}{G2}$$
\item A diagram showing modularisation of a quantum group at a root of unity:
$$\mathfig{0.3}{sl4-modularisation-domain} \qquad \text{or} \qquad \mathfig{0.3}{sl4-300}$$
\end{itemize}
We'd love to find an appropriate graphic illustrating string diagrams for a tensor category (possibly suggesting a Levin-Wen string net model), but haven't found anything.

\end{document}