\documentclass[11pt]{article}

%auto-ignore
%this ensures the arxiv doesn't try to start TeXing here.
%!TEX root =preproposal.tex

\usepackage{etex}
\usepackage{amsmath,amssymb,amsfonts,amsthm}
\usepackage{ifpdf}
\usepackage{leftidx}

\usepackage{etoolbox}
\usepackage{fullpage}
\usepackage{longtable}
\usepackage{pdflscape}
\usepackage{multicol}
\usepackage{enumerate}
\usepackage[all]{xy}
\input xy
\xyoption{all}

\usepackage{libertine}
\usepackage[T1]{fontenc}

\usepackage[backend=bibtex8,style=alphabetic,doi=false,isbn=false,url=false,minnames=6,maxnames=6]{biblatex}
\setcounter{biburlnumpenalty}{9000}
\setcounter{biburllcpenalty}{1000}
\setcounter{biburlucpenalty}{8000}
\renewbibmacro{in:}{%
  \ifentrytype{article}{}{\printtext{\bibstring{in}\intitlepunct}}}
\renewbibmacro*{volume+number+eid}{%
  \printtext{vol.}
  \printfield{volume}%
%  \setunit*{\adddot}% DELETED
  \setunit*{\addnbspace}% NEW (optional); there's also \addnbthinspace
  \printfield{number}%
  \setunit{\addcomma\space}%
  \printfield{eid}}
\DeclareFieldFormat[article]{number}{\mkbibparens{#1}}

\usepackage{xifthen}  % http://ctan.org/pkg/xifthen
\usepackage{breqn}
\usepackage{yfonts}
\usepackage{afterpage}

\usepackage{xcolor}
\definecolor{dark-red}{rgb}{0.7,0.25,0.25}
\definecolor{dark-blue}{rgb}{0.15,0.15,0.55}
\definecolor{medium-blue}{rgb}{0,0,0.65}
\definecolor{DarkGreen}{RGB}{0,150,0}


\ifpdf
\usepackage[pdftex,plainpages=false,hypertexnames=false,pdfpagelabels,breaklinks]{hyperref}
\else
\usepackage[dvips,plainpages=false,hypertexnames=false,breaklinks]{hyperref}
\fi
\hypersetup{
   colorlinks, linkcolor={purple},
   citecolor={medium-blue}, urlcolor={medium-blue}
}


% Page size %%%%%%%%%%%%%%%%%%%%%%%%%%%%%%%%%%%%%%%%%%%
\setlength\topmargin{0in}
\setlength\headheight{0in}
\setlength\headsep{0in}
\setlength\textheight{9in}
\addtolength{\hoffset}{-0.25in}
\addtolength{\textwidth}{.5in}
\setlength\parindent{0.25in}

\usepackage{tikz}
\usetikzlibrary{calc}
\usetikzlibrary{shapes}
\usetikzlibrary{backgrounds}
\usetikzlibrary{decorations.pathreplacing}
\usepackage{tikz-qtree}

\tikzstyle{shaded}=[fill=red!10!blue!20!gray!30!white]
\tikzstyle{unshaded}=[fill=white]
\tikzstyle{empty box}=[circle, draw, thick, fill=white, opaque, inner sep=2mm]
\tikzstyle{annular}=[scale=.7, inner sep=1mm, baseline]
\tikzstyle{rectangular}=[scale=.75, inner sep=1mm, baseline=-.1cm]



\usepackage{tcolorbox}
\tcbuselibrary{breakable}
\tcbuselibrary{skins}





% ----------------------------------------------------------------
\vfuzz2pt % Don't report over-full v-boxes if over-edge is small
\hfuzz2pt % Don't report over-full h-boxes if over-edge is small
% ----------------------------------------------------------------

% diagrams -------------------------------------------------------
% figures ---------------------------------------------------------
%%% borrowed from Dror's cobordisms paper, use this to include eps or pdf graphics.
\newcommand{\pathtodiagrams}{diagrams/}

\newcommand{\mathfig}[2]{{\hspace{-3pt}\begin{array}{c}%
  \raisebox{-2.5pt}{\includegraphics[width=#1\textwidth]{\pathtodiagrams #2}}%
\end{array}\hspace{-3pt}}}

\newcommand{\inputtikz}[1]{\input{diagrams/tikz/#1.tex}}

\newcommand{\arxiv}[1]{\href{https://arxiv.org/abs/#1}{\small  arXiv:#1}}
\newcommand{\arXiv}[1]{\href{https://arxiv.org/abs/#1}{\small  arXiv:#1}}
\newcommand{\doi}[1]{\href{https://dx.doi.org/#1}{{\small DOI:#1}}}
\newcommand{\euclid}[1]{\href{https://projecteuclid.org/getRecord?id=#1}{{\small  #1}}}
\newcommand{\mathscinet}[1]{\href{https://www.ams.org/mathscinet-getitem?mr=#1}{\small  #1}}
\newcommand{\googlebooks}[1]{(preview at \href{https://books.google.com/books?id=#1}{google books})}
\newcommand{\notfree}{}
\newcommand{\numdam}[1]{}

% THEOREMS -------------------------------------------------------
\theoremstyle{plain}
%\newtheorem*{fact}{Fact}
\newtheorem{prop}{Proposition}[section]
\newtheorem{conj}[prop]{Conjecture}
\newtheorem{thm}[prop]{Theorem}
\newtheorem{thmalpha}{Theorem}
\renewcommand*{\thethmalpha}{\Alph{thmalpha}}
\newtheorem{recipe}[prop]{Recipe}
\newtheorem{lem}[prop]{Lemma}
\newtheorem{cor}[prop]{Corollary}
\newtheorem{fact}[prop]{Fact}
\newtheorem{facts}[prop]{Facts}
\newtheorem*{cor*}{Corollary}
\newtheorem*{thm*}{Theorem}
%\newtheorem*{example}{Example}
\newtheorem{question}{Question}
\newenvironment{rem}{\\ \noindent\textsl{Remark.}}{}  % perhaps looks better than rem above?
\numberwithin{equation}{section}
%\numberwithin{figure}{section}

\theoremstyle{remark}
\newtheorem{example}[prop]{Example}
\newtheorem*{exc}{Exercise}
\newtheorem{remark}[prop]{Remark}           
\newtheorem*{rem*}{Remark}               %unnumbered remark
\newtheorem*{example*}{Example}                %unnumbered exercise

\theoremstyle{definition}
\newtheorem{defn}[prop]{Definition}         % numbered definition
\newtheorem{assumption}[prop]{Assumption}   
\newtheorem{nota}[prop]{Notation}   
\newtheorem*{defn*}{Definition}             % unnumbered definition

% \usepackage{parskip}
% Modifies the spacing above theorem environments, which is messed up when using the parskip package.
% (http://tex.stackexchange.com/questions/22119)
% \makeatletter \def\thm@space@setup{\thm@preskip=\parskip \thm@postskip=0pt} \makeatother

\newenvironment{mycolorbox}[1][]%
  {\if\detokenize{#1}\relax\relax%
      \begin{tcolorbox}%
    \else%
      \begin{tcolorbox}[#1]%
    \fi%
  \vspace{-3mm}%
  \parskip=0.5\baselineskip \advance\parskip by 0pt plus 2pt%
  \parindent=0pt%
}
  {\end{tcolorbox}}

\makeatletter
\@ifpackagelater{tcolorbox}{2015/01/01}%
  {%
    \newenvironment{boxedexample}{\begin{mycolorbox}[breakable,notitle,boxrule=1pt,colback=blue!5,colframe=blue!20,enhanced jigsaw]}{\end{mycolorbox}}
  }
  {%
    \newenvironment{boxedexample}{\begin{mycolorbox}[breakable,notitle,boxrule=1pt,colback=blue!5,colframe=blue!20]}{\end{mycolorbox}}
  }%
\makeatother
\newenvironment{boxedexample*}{\begin{mycolorbox}[notitle,boxrule=1pt,colback=blue!5,colframe=blue!20]}{\end{mycolorbox}}

%\usepackage{stmaryd}
\newcommand{\llbracket}{[\![}
\newcommand{\rrbracket}{]\!]}
\newcommand{\truth}[1]{\llbracket #1 \rrbracket}

\theoremstyle{plain}
\newcommand{\noqed}{\renewcommand{\qedsymbol}{}}

% Marginal notes in draft mode -----------------------------------
\newcommand{\scott}[1]{\stepcounter{comment}{{\color{green} $\star^{(\arabic{comment})}$}}\marginpar{\color{green}  $\star^{(\arabic{comment})}$ \usefont{T1}{scott}{m}{n}  #1 --S}}     % draft mode
\newcounter{comment}
\newcommand{\noop}[1]{}
\newcommand{\todo}[1]{\textcolor{blue}{\textbf{TODO: #1}}}
\newcommand{\nn}[1]{\textcolor{red}{[[#1]]}}

% \mathrlap -- a horizontal \smash--------------------------------
% For comparison, the existing overlap macros:
% \def\llap#1{\hbox to 0pt{\hss#1}}
% \def\rlap#1{\hbox to 0pt{#1\hss}}
\def\clap#1{\hbox to 0pt{\hss#1\hss}}
\def\mathllap{\mathpalette\mathllapinternal}
\def\mathrlap{\mathpalette\mathrlapinternal}
\def\mathclap{\mathpalette\mathclapinternal}
\def\mathllapinternal#1#2{%
\llap{$\mathsurround=0pt#1{#2}$}}
\def\mathrlapinternal#1#2{%
\rlap{$\mathsurround=0pt#1{#2}$}}
\def\mathclapinternal#1#2{%
\clap{$\mathsurround=0pt#1{#2}$}}

% MATH -----------------------------------------------------------
\newcommand{\Natural}{\mathbb N}
\newcommand{\Integer}{\mathbb Z}
\newcommand{\Rational}{\mathbb Q}
\newcommand{\Real}{\mathbb R}
\newcommand{\Complex}{\mathbb C}
\newcommand{\Field}{\mathbb F}

% tricky way to iterate macros over a list
\def\semicolon{;}
\def\applytolist#1{
    \expandafter\def\csname multi#1\endcsname##1{
        \def\multiack{##1}\ifx\multiack\semicolon
            \def\next{\relax}
        \else
            \csname #1\endcsname{##1}
            \def\next{\csname multi#1\endcsname}
        \fi
        \next}
    \csname multi#1\endcsname}

% \def\cA{{\cal A}} for A..Z
\def\calc#1{\expandafter\def\csname c#1\endcsname{{\mathcal #1}}}
\applytolist{calc}QWERTYUIOPLKJHGFDSAZXCVBNM;
% \def\bbA{{\mathbb A}} for A..Z
\def\bbc#1{\expandafter\def\csname bb#1\endcsname{{\mathbb #1}}}
\applytolist{bbc}QWERTYUIOPLKJHGFDSAZXCVBNM;
% \def\bfA{{\mathbf A}} for A..Z
\def\bfc#1{\expandafter\def\csname bf#1\endcsname{{\mathbf #1}}}
\applytolist{bfc}QWERTYUIOPLKJHGFDSAZXCVBNM;


\DeclareMathOperator{\depth}{depth}
\DeclareMathOperator{\nbhd}{nbhd}
\DeclareMathOperator{\Span}{span}
\DeclareMathOperator{\Tr}{Tr}
\DeclareMathOperator{\sh}{sh}
\DeclareMathOperator{\un}{un}
\DeclareMathOperator{\FPdim}{FPdim}
\DeclareMathOperator{\coeff}{coeff}

\newcommand{\set}[2]{\left\{#1\middle|#2\right\}}
\newcommand{\jw}[1]{f^{(#1)}}
\newcommand{\twoone}{{\rm II}$_1$}

\newcommand{\id}{\boldsymbol{1}}
\renewcommand{\imath}{\mathfrak{i}}
\renewcommand{\jmath}{\mathfrak{j}}

\newcommand{\qRing}{\Integer[q,q^{-1}]}
\newcommand{\qMod}{\qRing-\operatorname{Mod}}
\newcommand{\ZMod}{\Integer-\operatorname{Mod}}

\newcommand{\into}{\hookrightarrow}
\newcommand{\onto}{\mapsto}
\newcommand{\iso}{\cong}
\newcommand{\actsOn}{\circlearrowright}
\newcommand{\isoto}{\overset{\iso}{\to}}

\newcommand{\htpy}{\simeq}

\newcommand{\abs}[1]{\left|#1\right|}
\newcommand{\norm}[1]{\left|\left|#1\right|\right|}
\newcommand{\ip}[1]{\left< #1\right>}

\newcommand{\relations}[2]{\left<#1 \;\left| \; #2 \right. \right>}
\newcommand{\pairing}[2]{\left\langle#1 ,#2 \right\rangle}

\newcommand{\code}[1]{{\tt #1}}
\newcommand{\MMA}{\code{Mathematica} {}}

\makeatletter
\newcommand{\hashdef}[2]{\@namedef{#1}{#2}}
\newcommand{\hashlookup}[1]{\@nameuse{#1}}
\makeatother


\newcommand{\card}[1]{\sharp{#1}}

\newcommand{\bdy}{\partial}
\newcommand{\compose}{\circ}
\newcommand{\eset}{\emptyset}
\newcommand{\disj}{\sqcup}

\newcommand{\psmallmatrix}[1]{\left(\begin{smallmatrix} #1 \end{smallmatrix}\right)}

\newcommand{\directSum}{\oplus}
\newcommand{\DirectSum}{\bigoplus}
\newcommand{\tensor}{\otimes}
\newcommand{\Tensor}{\bigotimes}

\newcommand{\db}[1]{\left(\left(#1\right)\right)}

\newcommand{\su}[1]{\mathfrak{su}_{#1}}
\newcommand{\csl}[1]{\mathfrak{sl}_{#1}}
\newcommand{\uqsl}[1]{U_q\left(\csl{#1}\right)}


\newcommand{\Mat}[1]{\operatorname{\mathbf{Mat}}\left(#1\right)}
\newcommand{\Inv}[1]{\operatorname{Inv}\left(#1\right)}
\DeclareMathOperator{\Hom}{Hom}
%\newcommand{\Hom}[3]{\operatorname{Hom_{#1}}\!\!\left(#2 \to #3\right)}
\newcommand{\End}[1]{\operatorname{End}\left(#1\right)}
\newcommand{\im}{\operatorname{im}}
\newcommand{\Aut}{\operatorname{Aut}}
\newcommand{\Irr}{\operatorname{Irr}}
\newcommand{\Gal}{\operatorname{Gal}}

\newcommand{\lk}[2]{\operatorname{lk}\left(#1,#2\right)}
\newcommand{\fr}[1]{\operatorname{fr}\left(#1\right)}

\newcommand{\asbimod}[2]{\operatorname{Mod}'\left(#1 \subset #2\right)}
\newcommand{\sbimod}[2]{\operatorname{Mod}\left(#1 \subset #2\right)}
\newcommand{\abimod}[2]{#1-\operatorname{Mod}'-#2}
\newcommand{\bimod}[2]{#1-\operatorname{Mod}-#2}
\newcommand{\bimodule}[3]{\leftidx{_#1}{#2}{_#3}}

\newcommand{\pa}{\mathcal{PA}}
\newcommand{\TL}{\mathcal{TL}}
\newcommand{\JW}[1]{f^{(#1)}}
\newcommand{\tr}[1]{\text{tr}(#1)}
\newcommand{\dn}[1]{{\mathcal D}{\mathit (#1)}}
\newcommand{\Rep}{{\sf Rep}}
\newcommand{\gA}{{\textgoth{A}}}

\newcommand{\directSumStack}[2]{{\begin{matrix}#1 \\ \DirectSum \\#2\end{matrix}}}
\newcommand{\directSumStackThree}[3]{{\begin{matrix}#1 \\ \DirectSum \\#2 \\ \DirectSum \\#3\end{matrix}}}

\newcommand{\grading}[1]{{\color{blue}\{#1\}}}
\newcommand{\shift}[1]{\left[#1\right]}

\newcommand{\tensorover}[1]{\otimes_{#1}}
\newcommand{\tensorhat}{\widehat{\Tensor}}

\newcommand{\LL}{\mathcal{L}}
\newcommand{\Lhat}{\hat{\mathcal{L}}}
\newcommand{\writhe}{\operatorname{writhe}}

\newenvironment{narrow}[2]{%
\vspace{-0.4cm}% horrible hack, by scott % this only seems to be appropriate in beamer mode...
\begin{list}{}{%
\setlength{\topsep}{0pt}%
\setlength{\leftmargin}{#1}%
\setlength{\rightmargin}{#2}%
\setlength{\listparindent}{\parindent}%
\setlength{\itemindent}{\parindent}%
\setlength{\parsep}{\parskip}}%
\item[]}{\end{list}}

\makeatletter

%% this adds some diagnostic messages in the log file, helpful for tracking down permanent 'labels may have changed' warnings.
\def\@testdef #1#2#3{%
  \def\reserved@a{#3}\expandafter \ifx \csname #1@#2\endcsname
 \reserved@a  \else
\typeout{^^Jlabel #2 changed:^^J%
\meaning\reserved@a^^J%
\expandafter\meaning\csname #1@#2\endcsname^^J}%
\@tempswatrue \fi}


%%%Tikz macro

\newcommand{\roundNbox}[6]{
	\draw[rounded corners=5pt, very thick, #1] ($#2+(-#3,-#3)+(-#4,0)$) rectangle ($#2+(#3,#3)+(#5,0)$);
	\coordinate (ZZa) at ($#2+(-#4,0)$);
	\coordinate (ZZb) at ($#2+(#5,0)$);
	\node at ($1/2*(ZZa)+1/2*(ZZb)$) {#6};
}

\newcommand{\ncircle}[5]{
	\draw[thick, #1] #2 circle (#3);
	\node at #2 {#5};
	\filldraw[red] ($#2+(#4:#3cm)$) circle (.05cm);
%	\node at ($#2+(#4:.15cm)+(#4:#3cm)$) {$\star$};
}

\usetikzlibrary{decorations.pathmorphing}

\pgfdeclaredecoration{complete sines}{initial}
{
    \state{initial}[
        width=+0pt,
        next state=sine,
        persistent precomputation={\pgfmathsetmacro\matchinglength{
            \pgfdecoratedinputsegmentlength / int(\pgfdecoratedinputsegmentlength/\pgfdecorationsegmentlength)}
            \setlength{\pgfdecorationsegmentlength}{\matchinglength pt}
        }] {}
    \state{sine}[width=\pgfdecorationsegmentlength]{
        \pgfpathsine{\pgfpoint{0.25\pgfdecorationsegmentlength}{0.5\pgfdecorationsegmentamplitude}}
        \pgfpathcosine{\pgfpoint{0.25\pgfdecorationsegmentlength}{-0.5\pgfdecorationsegmentamplitude}}
        \pgfpathsine{\pgfpoint{0.25\pgfdecorationsegmentlength}{-0.5\pgfdecorationsegmentamplitude}}
        \pgfpathcosine{\pgfpoint{0.25\pgfdecorationsegmentlength}{0.5\pgfdecorationsegmentamplitude}}
}
    \state{final}{}
}

\title{Preproposal for MSRI semester on \textbf{Quantum Symmetries}}
\author{Scott Morrison and Noah Snyder}

\begin{document}
\maketitle

\nn{The sample preproposal is at \url{https://drive.google.com/file/d/1V3-aTmbB8uwoGgk5WAtz1kZ8cymrOPfLvqLozm_Ep0P_kl1ZmXQf6RBTm_6Q0oFnEQoVGxrys3Vs6VuO/view}.}

\nn{The recent proposal in algebraic topology is at \url{https://drive.google.com/file/d/0B3vGTOtp9YXBUXd5OEhFUV9FRUk/view?usp=sharing}}

\section{Quantum Symmetry}

\nn{
The description should include a background of the topic, relationships with other parts of mathematics or other disciplines, goals and areas of potential progress, an outline of the proposed program's structure, proposed involvement of postdocs and interactions with other scientists, and why the proposal is particularly suited to MSRI. It is highly desirable that the program show the breadth of the mathematical field being treated, and take into special account its relation to other fields of mathematics and (in appropriate cases) other sciences and engineering.
}

We are proposing a one-semester MSRI program with the goal of advancing and unifying the study of quantum symmetry throughout mathematics, with an emphasis on the appearance of tensor categories in the study of operator algebras, representation theory, mathematical physics, and topology.

\subsection{Organizers}
We propose the following primary organizers.
\begin{itemize}
  \setlength{\itemsep}{1pt}
  \setlength{\parskip}{0pt}
  \setlength{\parsep}{0pt}
\item Vaughan Jones, Vanderbilt University
\item Scott Morrison, Australian National University
\item Victor Ostrik, University of Oregon
\item Emily Peters, Loyola University Chicago
\item Eric Rowell, Texas A\&M
\item Noah Snyder, Indiana University
\end{itemize}
All are enthusiastic about the program, and can commit to attending for a substantial part of the semester. We have had successful past experiences organizing workshops with several subsets of this group.  We anticipate that both Scott Morrison and Noah Snyder will attend the entire program.

Emily Peters and \nn{} will organize the connections for women workshop, \nn{} will organize the introductory workshop, and \nn{} will organize the topical workshop.

% Potential additional organizers:
% Dietmar Bisch (Vanderbilt University)
% Yasuyuki Kawahigashi (Tokyo University)
% Alice Guionnet (MIT)
% Chelsea Walton (Temple University)

\subsection{What is Quantum Symmetry?}

Symmetry, as formalized by group theory, is ubiquitous across mathematics and science.  However, in some quantum settings, the notion of a group is no longer enough to capture all symmetries.  Important motivating examples include Galois-like symmetries of von Neumann algebras, anyonic particles in condensed matter physics, and deformations of universal enveloping algebras. It is not clear how to directly generalize the notion of a group to encompass these examples.

To achieve this generalization, we first take a different viewpoint on classical symmetries, replacing the group with an appropriate category of representations.  Such a category of representations has a rich structure: most crucially one can form tensor products of representations.  Thus to any group we can naturally assign a tensor category, which captures many of the features of the group. This replacement has a long history in both mathematics (pioneered by Weyl) and physics (where the basic excitations of a system are typically indexed by the irreducible representations of the symmetry group).

When we produce a tensor category as the representation theory of a group, the tensor product will always be commutative. We can consider the more general case, and thus begin the formalisation of all examples of quantum symmetries as various classes of tensor categories. This is a natural analogue of generalizing the notion of space in algebraic geometry, by replacing rings of functions on a variety with noncommutative rings, or of thinking of $C^*$-algebras as noncommutative topological spaces.

Quantum symmetry has appeared independently many times in mathematics in different guises, including tensor categories, topological phases of matter, subfactors, Hopf algebras, and quantum groups. This program thus represents a subject with broad connections across mathematics. Already, collaborations between experts coming from different origins have revealed deep connections between these subjects, but it seems clear that there is much more to come. This MSRI program will significantly expand the research connections between these fields, and lay the groundwork for a unified theory of quantum symmetry.

This is an exciting time for the field, with many important recent breakthroughs. Using higher Frobenius-Schur indicators, Bruillard-Ng-Rowell-Wang resolved a major open problem showing that there are only finitely many modular tensor categories of a given rank. Recently Bartlett-Douglas-Schommer-Pries-Vicary gave a complete classification of 3,2,1-dimensional topological field theories in terms of modular tensor categories. Significant progress has been made in the past few years on the classification of small index subfactors, extending the prior classification to index 4 all the way up to 5.25 and finding several new examples. Some of these examples are `quadratic categories', further studied by Izumi, and others appear truly exotic. These examples open the way to a rich new world of fusion categories which do not come from groups or quantum groups.  Despite all this progress, the field is still young and many important conjectures that remain open. Some notable examples include
\begin{itemize}
  \setlength{\itemsep}{1pt}
  \setlength{\parskip}{0pt}
  \setlength{\parsep}{0pt}
\item Can integral fusion categories  be classified group theoretically?
\item Are fusion categories always pivotal? (An analogue of Kaplansky's 5th conjecture.)
\item Does the dimension of any simple object divide the global dimension? (An analogue of Kaplansky's 6th conjecture.)
\item What is the complete classification of quadratic fusion categories?
\item Is every modular tensor category realized as the representation theory of a conformal field theory?
\end{itemize}
Such a decisive time in the development of the field makes it the perfect time for an MSRI program.

We are interested in developing general theoretical approaches to quantum symmetry, but equally important is the search for new examples. Historically the main sources of examples are classical groups and specializations of the Drinfeld-Jimbo quantum groups which deform the universal enveloping algebras of Lie algebras. However, several key examples appearing from subfactor classification results do not appear to come from these worlds. These unusual examples can in turn inform the general theory, since they allow one to distinguish between results that may hold in general and ones that just hold for groups and quantum groups for reasons special to those situations.

We want to particularly emphasize the rapid development of the study of quantum symmetries in condensed matter physics over the last decade. Physicists studying topological order have adopted the language of tensor categories, and their work suggests many new fundamental problems in the field. We intend that this semester program will include a substantial presence of physicists interested in topological order, and facilitate interactions between the mathematics and physics communities.

Tensor categories have a planar structure, with the category structure going in one direction and the tensor structure in the other direction. This has lead to Jones's notion of planar algebras in subfactor theory and also to \nn{other things}. More generally, there has been great recent interest in the broader subject of higher dimensional algebra, such as the study of $E_n$ algebras in $(\infty,1)$-categories. Tensor categories give some of the most accessible examples of this general theory.

\subsection{Areas of concentration}
Within the framework of studying the various guises of quantum symmetries, and their interactions,
we propose the following eight areas of focus for the semester.

\begin{enumerate}
  \setlength{\itemsep}{1pt}
  \setlength{\parskip}{0pt}
  \setlength{\parsep}{0pt}
\item Structure theory, constructions and classification of fusion categories and module categories over them, and applications to representation theory.
\item Structure theory of braided fusion categories, modular categories.
\item Nonsemisimple finite tensor categories, pointed Hopf algebras, and Nichols algebras.
\item Subfactors, planar algebras, constructions of tensor categories from them, and subfactors of small index.
\item Quantum invariants of knots and 3-manifolds, and local topological field theories.
\item Conformal nets, and vertex algebra constructions of tensor categories.
\item Topological order and topological quantum computation.
\item Deligne categories (interpolations of $Rep(S_n)$ and $Rep(GL_n)$ to complex values of $n$).
\end{enumerate}

We now briefly summarise the state of research in each area, highlighting connections and promising new directions.

\subsubsection{Fusion categories}
The category of finite dimensional complex representations of a finite group is semisimple and has only finitely many simple objects.  Fusion categories are tensor categories which share these properties and thus are quantum analogues of finite groups.

\todo{Noah}

\nn{ENO, graded extensions, solvable/nilpotent, small rank, }

\subsubsection{Braided fusion categories}
\todo{Noah}

\nn{Modular categories,
rank finiteness, computing centres of fusion categories, Frobenius-Schur indicators}

\subsubsection{Nonsemisimple finite tensor categories}
\todo{Noah}

\subsubsection{Subfactors and planar algebras}
To study a Galois field extension $K \subset M$, we begin by looking at the
Galois group of automorphisms of $M$ fixing $K$. For example, the intermediate
fields $K \subset L \subset M$ are parameterized by the subgroups of the
Galois group. This theory finds a noncommutative, or quantum, analogue in the
study of operator algebras, as the \emph{standard invariants of subfactors}.

The subject began with the striking observation by Jones that the index of a
subfactor was \emph{quantized}, and the possible values of the index are
exactly $\{4 \cos^2(\pi/n)\}_{n \geq 3} \cup [4,\infty]$. We now understand
that these restrictions are the consequence of a rich and rigid algebraic
structure underlying any subfactor subfactor.

Subfactors provide an alternative axiomatization of the notion of quantum
symmetry, especially adapted to the unitary setting. A subfactor is an
inclusion of von Neumann algebras with trivial centres. From a subfactor, one
can extract a finite algebraic gadget called the \emph{standard invariant}.
There have been many alternative ways to axiomatize and study these objects
(for example as Ocneanu's paragroups, Popa's $\lambda$-lattices, and Jones'
planar algebras). For our purposes here, a particularly interesting
alternative is as a pair $(\cC, A)$, where $\cC$ is a unitary pivotal fusion
category, and $A$ is an algebra object in $\cC$. The importance of subfactors
in the subject of quantum symmetries is seen in the fact that every such pair
in fact comes from some subfactor!

The techniques developed in the course of analyzing the standard invariants of
subfactors have resulted in unusual constructions of examples as well as
powerful obstructions to the existence of certain families. Indeed, the most
exotic known fusion categories at present --- the categories of $A-A$ or $B-B$
bimodules, for $A \subset B$ the extended Haagerup subfactor --- apparently
unobtainable starting from finite groups or quantum groups, have been
constructed as part of the program of classifying small index subfactors.

The various alternative axiomatizations of a standard invariant emphasize
different aspects of the algebraic structure, and typically suggest rather
different approaches. As an example, the `jellyfish algorithm', essential in
the construction of the extended Haagerup subfactor, is very natural from the
perspective of $\lambda$-lattices (although this relationship was only
understood later!). Further, the planar algebra approach emphasizes the
importance of the representation theory of the annular Temperley-Lieb-Jones
category, and this has led to very powerful constraints on quantum symmetries
which would not be obvious from a more algebraic approach. It seems clear that
these alternative perspectives have not been exhausted, and indeed will have
impact throughout the study of quantum symmetries.

Recent highlights of research into subfactors with consequences for the broader understanding of quantum symmetries include:
\begin{itemize}
\item The classification of subfactors to progressively higher and higher indices (these results can be effortlessly translated in classifications of algebra objects). In particular, these classifications have produced unexpected and very poorly understood examples, and further, provided unxpected evidence that high `supertransitivity' is very rare (tantalizingly analogous to the fact that highly transitive group actions are rare).
\item Amongst the examples discovered during the classification program have been several fusion categories now considered as part of the family of near group fusion categories, or more generally quadratic fusion categories. Quite a lot is now known, especially by work of Izumi and Evans-Gannon, and many more examples have been constructed. The complete theory has not been achieved, but may be close.
\item \nn{the non-amenable world ... }
\item \nn{intermediate subfactors}
\item \nn{Brauer-Picard groupoids}
\end{itemize}


\subsubsection{Quantum invariants and topological field theory}
The connection between quantum invariants of knots and quantum symmetries goes right back to the beginnings of both subjects. Jones' discovery of his knot polynomial arose through studying the structure of the standard invariant of a subfactor. He
realized that the Temperley-Lieb algebras are always present inside the tower of relative commutants of a subfactor (one of the earliest views of the standard invariant of a subfactor),
and subsequently realized the there are homomorphisms from the braid groups into the Temperley-Lieb algebras which can be used to build the Jones polynomial.

Since then, the Jones polynomial has been massively generalized, and we now understand that any object in a braided pivotal category \nn{different adjectives?} gives rise to a knot invariant --- the original case of the Jones polynomial comes from the standard representation of $U_q \mathfrak{sl}_2$.

Moreover, beginning with the work of Reshetikhin-Turaev, and many others, leading up to the recent work of Lurie, we understand a deep relationship between higher categories and  topological field theories, now expressed via the cobordism hypothesis. Higher categories with appropriate invariance conditions give rise to local topological field theories. Local topological field theories in $(d+1)$-dimensions assign algebraic data to all manifolds of degree at most $d$, and the algebraic data associated to a point recovers a higher category. The cobordism hypothesis explains in detail how these constructions give a correspondence. The broadest definition of a quantum symmetry should be exactly those higher categories that appear in some version of this correspondence.

Recent work in particular dimensions, and working in particular algebraic contexts, has pinned down precisely the relevant higher categories. For $(2+1)$-dimensional local field theories with values in tensor categories, the values at a point (that is, the possible quantum symmetries) are precisely fusion categories with nonzero global dimension. There is still a great deal of work
to be done understanding what the cobordism hypothesis tells us in each dimension, and in other contexts. It seems likely that answers to these questions will give extremely profound indications of the most important classes of quantum symmetries, and will direct much of future research in the field.

\subsubsection{Conformal field theory}
\todo{Scott}

The connection between conformal field theory, and quantum symmetry as introduced here, is far from well understood. Nevertheless, an essential connection does exist, and there are many tantalizing hints of its importance.

Certainly, the representation category of a rational conformal field theory gives a modular tensor category. It is possible, and indeed strongly suspected by some, that in fact every modular tensor category is realized in this way.
Thus, as is the case with subfactors, conformal field theory can be thought of
as a host (perhaps universal) for certain types of quantum symmetries. Many early examples of quantum symmetries (for example, the 2221 fusion category) were first constructed via conformal field theory, although in many cases subsequently a purely algebraic approach has been developed --- often enough
transporting some construction earlier known in CFT to the categorical
setting. It is very likely that CFT constructions will inspire further
development of structure theory for fusion categories and modular tensor
categories.

There are alternative axiomatizations of conformal field theories, e.g.  as
vertex operator algebras or as conformal nets (although the precise
relationship between them is still being worked out). The conformal net
approach is intimately related with the theory of subfactors. \nn{... and?
...}

\nn{conformal nets as a target for TFT}

\subsubsection{Topological order}
\todo{Scott}

The discovery of topological order in condensed matter physics is a
fundamental theoretical breakthrough --- and hopefully in the future, with
some engineering breakthroughs, the foundation of new technologies including
topological quantum computation. Quantum symmetries promise to provide the
mathematical foundations for this physics and engineering.

Topological order goes beyond the standard theory of symmetry breaking, going
back to Landau, and can be understood in terms of systems whose `symmetries'
are not classical symmetries described by groups, but rather quantum
symmetries as discussed here. In particular, the fascinating realization is
that there are certain physical systems with effective descriptions via local
topological field theories. For physicists, topological order can be
characterized by the presence of robust ground state degeneracies, and these
degeneracies are exactly the vector space assigned by the governing
$d+1$-dimensional topological field theory to the $d$-manifold describing the
shape of the system. The higher categorical data determining the local
topological field theory should be thought of as the quantum symmetry of the
topological phase. In particular, for $2+1$-dimensional topological field
theory, this data is precisely a fusion category, as discussed above.



\nn{symmetry protected phases, Dijgraaf-Witten TFTs, fractional quantum Hall effect, Levin-Wen and Walker-Wang models, topological quantum computing}

\subsubsection{Deligne categories}
The representation theory of the classical families of Lie Groups $GL_n$, $O_n$, and $SP_n$, turn out to behave uniformly in $n$. For example, the dimension of a representation of $GL_n$ whose highest weight is supported on one end of the $A_{n+1}$ Dynkin diagram has dimension given by a polynomial in $n$. Deligne and Milne realized that these patterns could be made more precise: there is a $1$-parameter family of symmetric tensor categories $\mathrm{Rep}(GL_t)$, such that when $t=n$ the semisimplification of this category recovers $\mathrm{Rep}(GL_n)$. Similar ideas were developed also by Cvitanovic and Vogel. The HOMFLY and Kauffman skein theories can be thought of as quantum versions of $GL_t$ and $OSP_t$.  Deligne also developed a similar family $\mathrm{Rep}(S_t)$ which interpolates between the categories of representations of the symmetric groups. There is some tantalizing evidence due to Deligne-Gross, Vogel, Cohen-de Mann, and others that suggests that there may be a similar $1$-parameter family of symmetric tensor categories which interpolates between the exceptional lie algebras.

\section{Related Programs}
\nn{The proposer should include information on recent and planned programs in the area held at other Institutes and Universities. Have there been related programs at MSRI?}




\paragraph{Programs:}
\begin{itemize}
  \setlength{\itemsep}{1pt}
  \setlength{\parskip}{0pt}
  \setlength{\parsep}{0pt}
\item Subfactors and their applications \href{https://www.newton.ac.uk/event/oas}{program} (Isaac Newton Institute, early 2017)
\item Von Neumann algebras \href{https://www.him.uni-bonn.de/programs/future-programs/future-trimester-programs/von-neumann-algebras-2016/description/}{trimester} (Hausdorff Institute for Mathematics, early 2016)
\item Homotopy theory, fields, and manifolds \href{https://www.him.uni-bonn.de/programs/past-programs/past-trimester-programs/homotopy-theory-2015/description/}{trimester} (Hausdorff Institute for Mathematics, early 2015)
\end{itemize}

There have not been any recent related programs at MSRI. The 2000-2001 program on operator algebras studied subfactors and their standard invariants. Of course, the famous parallel programs in low-dimensional topology and in operators algebras during 1984-1985 led directly to much of the fantastic interplay between topology and quantum symmetry described in this proposal!
% cf Vaughan's remarks in https://math.berkeley.edu/~vfr/msri.pdf

\paragraph{Conferences:}
\begin{itemize}
  \setlength{\itemsep}{1pt}
  \setlength{\parskip}{0pt}
  \setlength{\parsep}{0pt}
\item Modular categories (Oaxaca, Mexico, August 2016)
\item Nichols Algebras and Their Interactions with Lie Theory, Hopf Algebras and Tensor Categories (Banff, September 2015)
\item Factorizable structures in algebra and geometry (Banff, August 2015)
\item Subfactor Theory in Mathematics and Physics (Qinhuangdao, China, July 2015)
\item Subfactors and Conformal Field Theory (Oberwolfach, March 2015)
\item Symmetry and Topology in Quantum Matter (IPAM, January 2015)
\item Subfactors and fusion categories (Banff, April 2014)
\item Fusion categories (Dijon, May 2013)
\item Classifying fusion categories (AIM, March 2012)
\item Subfactors in Maui (Hawaii, July 2011/2012/2013/2014)
\end{itemize}

\section{Key participants}
\nn{Identify key possible participants and their affiliations. A program should contain a significant component for postdoctoral fellows, including facilitating access to long-term senior visitors.}
See separate spreadsheet: Potential MSRI participants

We indicate possible participants below, although it is far from being an exhaustive list. We've annotated each potential participant according to their interests in the topics 1-8 described above, to the best of our knowledge, and also marked researchers who have already indicated their interest in attending a semester program to us with an asterisk.

\begin{multicols}{2}
\begin{itemize}
  \setlength{\itemsep}{1pt}
  \setlength{\parskip}{0pt}
  \setlength{\parsep}{0pt}
\item Nicolas Andruskiewitsch (Cordoba), [3]*
\item Iv\'an Angiono (Cordoba), [3]*
\item David Ayala (Montana State), [5]*
\item Dietmar Bisch (Vanderbilt), [4]*
\item Fiona Burnell (University of Minnesota), [1,7]*
\item Jonathan Comes (College of Idaho), [8]
\item Alexei Davydov (Ohio), [1,2,3]*
\item Chris Douglas (Oxford), [5,6]*
\item Inna Entova (MIT), [8]*
\item Pavel Etingof (MIT), [1,2,3,8]*
\item John Francis (Northwestern), [5]*
\item Dan Freed (UT Austin), [5,7]*
\item Michael Freedman (Microsoft Station Q), [5,7]
\item Jurgen Fuchs (Karlstads Universitet), [5,6]
\item Terry Gannon (Alberta), [1,4,6]
\item Shlomo Gelaki (Technion), [1,2,3]
\item Pinhas Grossman (UNSW), [1,4]*
\item Matthew Hastings (Microsoft Station Q), [7]
\item Istvan Heckenberger (Marburg), [3]
\item Andre Henriques (Utrecht), [1,4,5,6]*
\item Masaki Izumi (Kyoto), [4]
\item Theo Johnson-Freed (Northwestern), [1,4,5,8]*
\item Vaughan Jones (Vanderbilt), [4,6]*
\item David Jordan (Edinburgh), [1,5]*
\item Anton Kapustin (Caltech), [5,7]
\item Yasuyuki Kawahigashi (Tokyo), [4,5,6]*
\item Alexander Kirillov (Stony Brook), [5]
\item Alexei Kitaev (Caltech), [5,7]
\item Greg Kuperberg (Davis), [1,3,5,7]*
\item Michael Levin (University of Chicago), [7]
\item Jacob Lurie (Harvard), [5]
\item Scott Morrison (ANU), [1,2,4,5,7,8]
\item Michael M\"uger (Nijmegen), [1,2,4]*
\item Sonia Natale (Cordoba), [3]
\item Richard Ng (Louisiana State), [1,2]
\item Dmitri Nikshych (New Hampshire), [1,2,3]*
\item Victor Ostrik (Oregon), [1,2,3,8]*
\item Emily Peters (Loyola University Chicago), [1,4]*
\item Julia Pevstova (Washington), [1,3]*
\item Julia Plavnik (Texas A\&M), [1,2]*
\item Nicolai Reshetikhin (Berkeley), [5]*
\item Eric Rowell (Texas A\&M), [1,2,7]*
\item Peter Schauenburg (Munich), [1,2]
\item Claudia Scheimbauer (MPIM Bonn), [5]
\item Hans-J\"urgen Schneider (Munich), [3]
\item Chris Schommer-Pries (MPIM Bonn), [1,5]
\item Christoph Schweigert (Hamburg), [5,6]*
\item Noah Snyder (Indiana), [1,3,4,5,8]*
\item Stephan Stolz (Notre Dame), [5,6]
\item Peter Teichner (MPIM/Berkeley), [1,5,6]
\item Vladimir Turaev (Indiana), [5]
\item Monica Vazirani (Davis), [5]*
\item Kevin Walker (Microsoft Station Q), [5,7]*
\item Chelsea Walton (Temple), [3]*
\item Zhenghan Wang (Microsoft Station Q), [2,7]*
\item Xiao-Gang Wen (MIT), [7]
\item Sarah Witherspoon (Texas A\&M), [3]*
\end{itemize}
\end{multicols}

As the planning horizon is still quite advanced, it is impossible to know who will be the best candidates for postdoctoral fellows during the program. Nevertheless, as the postdoctoral fellows form an extremely important part of the program, we will remain in close contact with all of the key participants of the program, asking them to advertise the possibility of postdoctoral positions, and to alert us to promising candidates. As many fields involved in the proposal are new and fast moving, we hope that junior researchers will play an active role in the program, including as speakers during all three of the workshops. We will coordinate with the postdoctoral fellows to organize regular seminar series at MSRI, both for the the postdocs to talk about their own research, and to ensure close cooperation between the postdocs, short term visitors, and long term visitors. As the event comes closer and we have indications of the times that various participants will be present, we will ensure that special sequences of MSRI seminars coincide with particular periods that sub-fields are strongly represented. 

\section{Workshops}
The program will begin with a Connection for Women workshop, to be organised by Emily Peters; other possible organisers include Chelsea Watson, Claudia Scheimbauer, and Fiona Burnell. \nn{more detail?}

For the introductory workshop, we suggest having one speaker for each of the 8 streams of the conference, each giving two talks. The speakers will be chosen as the best expository speakers,
able to give clear and motivated introductions to a broad audience. Possible speakers include
\begin{itemize}
  \setlength{\itemsep}{1pt}
  \setlength{\parskip}{0pt}
  \setlength{\parsep}{0pt}
\item Victor Ostrik (fusion categories) 
\item Eric Rowell (modular tensor categories)
\item Dmitri Nikshych (nonsemisimple finite tensor categories) % alternatives: Victor
\item Emily Peters (subfactors and planar algebras) % alternatives: Dave Penneys, 
\item Chris Schommer-Pries (local field theories) % alternatives: Noah, Chris D, Teichner
\item Andre Henriques (conformal nets) % alternatives: Terry Gannon, Chris Douglas
\item Zhenghan Wang (topological order and quantum computing) % alternatives: Eric Rowell, physicists?
\item Pavel Etingof (Deligne categories)
\end{itemize}
and closer to the date appropriate junior researchers will be invited. 

We propose that the topical workshop be on \emph{Tensor categories and topological quantum field theories}. These two subjects are quite closely intertwined, and it was the relationship between tensor categories and Turaev-Viro and Witten-Reshetikhin-Turaev topological field theories which drove much of the study of tensor categories in its early days.  Recently, the development of extended topological field theories has revitalized these two fields and made the connection between them more explicit. In one direction, topological field theory gives topological approaches to studying tensor categories. In the other direction, tensor categories give some of the most accessible examples of topological field theories.

Potential organizers would be David Ben-Zvi, John Francis, David Jordan, Scott Morrison and Chris Schommer-Pries. % possible alternatives: Chris Douglas, Claudia Scheimbauer, David Ayala.
\nn{Say something about how this fits with the entire program.}
Potential speakers would include \nn{work on this list}
\begin{multicols}{2}
\begin{itemize}
  \setlength{\itemsep}{1pt}
  \setlength{\parskip}{0pt}
  \setlength{\parsep}{0pt}
\item David Ayala	(Montana State)
\item Chris Douglas	(Oxford)
\item Dan Freed	(UT Austin)
\item Anton Kapustin	(Caltech)
\item Alexander Kirillov	(Stony Brook)
\item Claudia Scheimbauer	(MPIM Bonn)
\item Chris Schommer-Pries	(MPIM Bonn)
\item Stephan Stolz	(Notre Dame)
\item Peter Teichner	(MPIM/Berkeley)
\item Theo Johnson-Freed	(Northwestern)
\item Vladimir Turaev	(Indiana)
\item Kevin Walker (Microsoft)
\item Jacob Lurie (Harvard)
\end{itemize}
\end{multicols}


\end{document}